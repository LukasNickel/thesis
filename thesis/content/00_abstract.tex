\thispagestyle{plain}

\section*{Kurzfassung}
\begin{german}
In den letzten Jahrzehnten hat die $\gamma$-Astronomie 
viele Erkenntnisse über die Zusammensetzung des Universums gebracht.
Mitverantwortlich dafür waren die großen Image Air Cherenkov Telescope (IACT)
Experimente der dritten Generation.
In Zukunft soll diese Forschung begleitet und erweitert werden 
durch ein Experiment der nächsten Generation.
Das Cherenkov-Teleskope-Array (CTA) wird die Sensitiviät weiter steigern und den
beobachtbaren Energiebereich erweitern.
In dieser Arbeit werden Rekonstruktionsalgorithmen für CTA vorgestellt und Studien auf 
Monte-Carlo-Daten durchgeführt mit dem Fokus auf der frühen Phase des Experimentes.
Zu Beginn werden nur wenige Teleskope errichtet sein, sodass Ereignisse nur 
mit wenigen Teleskopen gesehen werden.
\end{german}

\section*{Abstract}
\begin{english}
In the past years the $\gamma$-astronomy has lead to many interesting
insights about the fundamental processes of the universe.
Part of that were the big imaging air cherenkov telescopes (IACTs) 
of the third generation.
In the future this research will be expanded with a next generation experiment.
The Cherenkov Telescope Array (CTA) will improve the sensitivity and
expand the observable energy range.
In this work we will present reconstruction algorithms for CTA und perform 
studies on monte carlo data with the focus on the early stages of the experiment.
In this stages CTA will only have some of the planned telescopes installed
leading to low multiplicity events.
\end{english}
