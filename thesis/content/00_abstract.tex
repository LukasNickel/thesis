\thispagestyle{plain}

\section*{Abstract}
%\begin{english}
In the past years, gamma ray astronomy has come a long way
to now be one of the main pillars of astronomy and astroparticle physics.
Many interesting insights have been found by the large Imaging Air Cherenkov Telescopes (IACTs) 
of the third generation, like MAGIC and HESS.

In the future, this research will be expanded with a next generation experiment.
The Cherenkov Telescope Array (CTA) will improve the sensitivity and
expand the observable energy range with 
three different types of telescopes and two telescope arrays, one on the northern and 
one on the southern hemisphere.
Proposed in 2005, the first telescopes for CTA are currently being tested.
CTA will have its own processing pipeline, called ctapipe, which is
actively being developed and released under an open-source license.
The experiment itself is to be operated as an open observatory.

In this work, I present reconstruction algorithms for CTA and perform 
studies on monte carlo data focusing on the reconstruction of the source position.
The analysis is based on the use of ctapipe and the
aict-tools package, which handles machine learning for IACTs.
In the early stages, CTA will only have some of the planned telescopes installed,
which means that showers will be observed by fewer telescopes.
As will be worked out, these low multiplicity events limit the stereoscopic
performance with the default geometric approach.

An alternative, based on the stereoscopic DISP-reconstruction employed by
the MAGIC-experiment, is evaluated in this thesis.
Favorable results are obtained for low energy and low multiplicity
events, whereas
high energy and high multiplicity events favor the geometric approach.
%\end{english}
