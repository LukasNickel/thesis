\thispagestyle{plain}

\iffalse
\section*{Kurzfassung}
\begin{otherlanguage}{german}
In den letzten Jahrzehnten hat die $\gamma$-Astronomie 
viele Erkenntnisse über die Zusammensetzung des Universums gebracht.
Mitverantwortlich dafür waren die großen Image Air Cherenkov Telescope (IACT)
Experimente der dritten Generation.
In Zukunft soll diese Forschung begleitet und erweitert werden 
durch ein Experiment der nächsten Generation.
Das Cherenkov-Teleskope-Array (CTA) wird die Sensitiviät weiter steigern und den
beobachtbaren Energiebereich erweitern.
In dieser Arbeit werden Rekonstruktionsalgorithmen für CTA vorgestellt und Studien auf 
Monte-Carlo-Daten durchgeführt mit dem Fokus auf der frühen Phase des Experimentes.
Zu Beginn werden nur wenige Teleskope errichtet sein, sodass Ereignisse nur 
mit wenigen Teleskopen gesehen werden.
\end{otherlanguage}
\fi

\section*{Abstract}
%\begin{english}
In the past years $\gamma$-ray astronomy has come a long way
to now be one of the main pillars of astronomy and astroparticle physics.
Many interesting insights have been found by the big imaging air cherenkov telescopes (IACTs) 
of the third generation.

In the future this research will be expanded with a next generation experiment.
The Cherenkov Telescope Array (CTA) will improve the sensitivity and
expand the observable energy range with 
three different types of telescopes and two huge arrays on the northern and 
southern hemisphere.
Proposed in 2005, the first telescopes for CTA are currently being tested.
CTA will have its own processing pipeline, called ctapipe, which is
actively being developed and released under an open-source license.

In this work we will present reconstruction algorithms for CTA and perform 
studies on monte carlo data focussing on the reconstruction of the source position.
Our analysis will be based on the use of ctapipe and 
aict-tools, another publicly available software package for IACT experiments.
In the early stages CTA will only have some of the planned telescopes installed,
which means that showers will be observed by less telescopes.
As we shall see, these low multiplicity events turn out to be problematic for
the HillasReconstructor,
the default geometrical reconstruction algorithm in ctapipe.

An alternative, based on the stereoscopic DISP-reconstruction employed by
the MAGIC-experiment, is being tested.
Favorable results are obtained in the low energy region whereas the
high energy and high event multiplicity region favors the HillasReconstructor.
%\end{english}
