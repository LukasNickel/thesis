\chapter{Gamma Astronomy V0.0}
\section{astro allgemein V0.1}
kurz halten, wenig aus dem bisherigen nehmen
kurze erwähnung von multi messenger, aber neutrinos und 
grav waves juckt uns eh nicht.
gammas pointen zur quelle, hadrons nicht
wir machen gammas!
ohne viel history
\section{gamma research, interessante physik V0.0}
\section{gamma entstehung V0.1}
entstehung gammas. mit kleinen erweiterungen schon da.
kurz: hadrons gibts auch. ist für uns background, weil mans auch misst
und es viel mehr gibt. gleichzeitig für den fall nicht so interessant.
\section{types of experiments, vor und nachteile V0.1}
direkt vs indirekt.
satelliten, tanks und iacts -> unterschiede/vorteile/nachteile.
Wir machen indirekt mit aicts.
\subsection{satelites V0.1}
\subsection{tanks V0.1}
\subsection{iacts V0.1}
\section{iacts V0.1}
\subsection{cherenkov licht, messen am boden V0.1}
\subsection{gamma shower V0.1}
\subsection{hadron shower V0.1}

\chapter{CTA V0.1}
sensitivity relativ zum crab für alle + cta angeben 
lst,mst,sst plots vielleicht text drum wrappen?
\section{current gen V0.1}
wir haben aktuell diese experimente:

sensitivity als crab flux ergänzen überall
\subsection{magic V0.1}
\subsection{veritas V0.1}
\subsection{hess V0.1}
\section{wie viel besser wird cta V0.1}
vergleich zu vorher: arrays auf beiden hemisphären, verschiedene teleskope, viele teleskope
status cta aktuell.
cta besonderheiten.
auflösungsrequirements unterbringen!
\subsection{lst V0.1}
passt grob
\subsection{mst V0.1}
passt grob
\subsection{sst V0.1}
passt grob

\chapter{Methods: Relativ allgemein V0.1}
wie werden daten genommen?
\section{Coordinate Frames V0.1}
wichtig, vor allem altaz, camera, nominal, ground
\section{Monte carlo V0.1}
ist wichtig duh
\subsection{corsika V0.1}
atmosphäre
\subsection{simtel V0.0}
telescope response
\section{ctapipe V0.1}
was ist das, warum ist das und was habe ich gemacht?
-> neu: num islands, fact cleaning
\subsection{telescope level  V0.1}
damit wir damit arbeiten können, muss erstmal kram gemacht werden, den
wir mal preprocessing nennen.
ctapipe ist nicht fertig, daher machen wir eigenen kram -> kai auf basis von ctapipe
liste der steps im detail.
\subsection{array level V0.1}
das sind so sachen wie energie, typ, richtung.
\section{Additional apporaches V0.1}
\subsection{disp methode, mono wie stereo V0.1}
\subsection{classification V0.1}
\subsection{machine learning V0.1}


\chapter{Analysis: Methods spezifiziert für unser Problem! V0.0}
mit zuvor erklärten methoden zusammen fassen was getan wird
\section{preprocessing  and used data V0.1}
erhaltene features hier oder dadrunter?
\subsection{telescope level V0.1}
\subsection{array level V0.1}
\section{high level alternative V0.0}
\subsection{gamma hadron sep V0.1}
machen wir auch per random forest 
\subsection{DISP V0.1}
random forest
median und iterative
\subsection{mono modell? V0.0}
wenn zeit ist als vergleich bei multi 2\&3
-> vlt machen die core und hmax schätzungen das da tatsächlich schlechter?

\chapter{Results V0.0}
\section{gamma/hadron V0.0}
\section{position mono V0.0}
\section{position stereo V0.0}
\section{position stereo cut V0.0}
\section{position stereo ohne hillas features, optimized for multi 2 ? V0.0}

\chapter{Conclusion V0.0}
