\chapter{Introduction V0.0}
\section{astro V0.0}
kurz halten, wenig aus dem bisherigen nehmen
kurze erwähnung von multi messenger, aber neutrinos und 
grav waves juckt uns eh nicht.
gammas pointen zur quelle, hadrons nicht
wir machen gammas!
ohne viel history
\section{gamma V0.0}
\subsection{physik, die uns interessiert V0.0}
quellen, offene physik fragen in verbindung mit gamma rays.
kurz
\subsection{gamma signal, hadron background V0.0}
entstehung gammas. mit kleinen erweiterungen schon da.
kurz: hadrons gibts auch. ist für uns background, weil mans auch misst
und es viel mehr gibt. gleichzeitig für den fall nicht so interessant.
\subsection{types of experiments, vor und nachteile V0.0}
direkt vs indirekt.
satelliten, tanks und iacts -> unterschiede/vorteile/nachteile.
Wir machen indirekt mit aicts.
\subsection{iacts V0.0}
entspricht quasi 1.3 von vorher
\subsubsection{gamma shower V0.0}
passt grob
\subsubsection{hadron shower V0.0}
mini anpassungen
\subsubsection{cherenkov licht, messen am boden V0.0}
das meiste kann eigentlich weg, da in methods erklärt.

\chapter{CTA V0.0}
sensitivity relativ zum crab für alle + cta angeben 
lst,mst,sst plots vielleicht text drum wrappen?
\section{current gen V0.0}
wir haben aktuell diese experimente:
\subsection{magic V0.0}
\subsection{veritas V0.0}
\subsection{hess V0.0}
\section{wie viel besser wird cta V0.0}
vergleich zu vorher: arrays auf beiden hemisphären, verschiedene teleskope, viele teleskope
status cta aktuell.
cta besonderheiten.
auflösungsrequirements unterbringen!
\subsection{lst V0.0}
passt grob
\subsection{mst V0.0}
passt grob
\subsection{sst V0.0}
passt grob

\chapter{Methods: Relativ allgemein V0.0}
wie werden daten genommen?
what needs to be done in an analysis?
we want to simulate stuff to do supervised learning and control algorithms.
\section{Coordinate Frames}
wichtig, vor allem altaz, camera, nominal, ground
\section{Monte carlo V0.0}
ist wichtig duh
\subsection{corsika V0.0}
atmosphäre
\subsection{simtel V0.0}
telescope response
\section{ctapipe V0.0}
was ist das, warum ist das und was habe ich gemacht?
-> neu: num islands, fact cleaning
\subsection{"preprocessing" V0.0}
damit wir damit arbeiten können, muss erstmal kram gemacht werden, den
wir mal preprocessing nennen.
ctapipe ist nicht fertig, daher machen wir eigenen kram -> kai auf basis von ctapipe
liste der steps im detail.
\subsection{high level V0.0}
das sind so sachen wie energie, typ, richtung.
das kann man teilweise analytisch machen oder
per machine learning. wir nutzen nachher random forests 
\subsection{hillas reconstructor V0.0}
\section{ersatz für missing ctapipe stuff / alternative}
\subsection{disp methode, mono wie stereo V0.0}
\subsection{machine learning V0.0}
etwas angepasst das, was ich dazu schon habe.
aict tools nennen.


\chapter{Analysis: Methods spezifiziert für unser Problem! V0.0}
mit zuvor erklärten methoden zusammen fassen was getan wird
\section{preprocessing auflisten V0.0}
erhaltene features hier oder dadrunter?
\section{high level auflisten V0.0}
\subsection{gamma hadron sep V0.0}
machen wir auch per random forest 
\subsection{mono V0.0}
random forest
\subsection{stereo V0.0}
median und iterative
\subsection{cut V0.0}
ist nicht so pralle das so zu machen, aber naja
\subsection{mono modell? V0.0}
wenn zeit ist als vergleich bei multi 2&3
-> vlt machen die core und hmax schätzungen das da tatsächlich schlechter?

\chapter{Results V0.0}
\section{gamma/hadron V0.0}
\section{position mono V0.0}
\section{position stereo V0.0}
\section{position stereo cut V0.0}
\section{position stereo ohne hillas features, optimized for multi 2 ? V0.0}

\chapter{Conclusion V0.0}