\thispagestyle{empty}
\setcounter{page}{2}
\section*{Hinweise}
Empfohlen wird die Verwendung dieser Vorlage mit der jeweils aktuellsten TeXLive Version (Linux, Windows) bzw. MacTeX Version (MacOS).
Aktuell ist dies TeXLive 2016. Download hier:
\begin{center}
  \ttfamily\url{https://www.tug.org/texlive/}
\end{center}
Bei Verwendung von TexLive Versionen 2014 und älter sollte
die Zeile
\begin{center}
\verb+\RequirePackage{fixltx2e}+ 
\end{center}
als erste Zeile der Präambel noch vor der Dokumentenklasse eingefügt werden.
Dies lädt diverse Bugfixes für LaTeX, die ab TexLive 2015 Standard sind.

Die Vorlage \texttt{thesis.tex} ist für die Kompilierung mit \texttt{lualatex} ausgelegt, mit wenigen Anpassungen kann sie aber auch mit \texttt{pdflatex} oder \texttt{xelatex} verwendet werden. 
Die Dokumentenklasse \texttt{tudothesis.cls} kann mit allen drei Programmen verwednet werden.

Achten Sie auch auf die Kodierung der Quelldateien.
Bei Verwendung von Xe\LaTeX\ oder Lua\LaTeX\ (empfohlen) müssen die
Quelldateien UTF-8 kodiert sein.
Bei Verwendung von pdf\LaTeX\ nutzen Sie die Pakete \texttt{inputenc} und \texttt{fontenc} mit der korrekten Wahl der Kodierungen.

Eine aktuelle Version dieser Vorlage steht unter 
\begin{center}
  \ttfamily\url{https://github.com/maxnoe/tudothesis}
\end{center}
zur Verfügung.

Alle verwendeten Pakete werden im \LaTeX{} Kurs von Pep et al.\ erklärt:
\begin{center}
  \ttfamily\url{http://toolbox.pep-dortmund.org/notes}
\end{center}

Für Rückmeldungen und bei Problemen mit der Klasse oder Vorlage, bitte ein \emph{Issue} auf GitHub aufmachen oder eine Email an
\href{mailto:maximilian.noethe@tu-dortmund.de}{maximilian.noethe@tu-dortmund.de} schreiben.

Wenn Sie die Dokumentenklasse mit der Option \texttt{tucolor} laden, werden verschiedene Elemente in TU-Grün gesetzt.
