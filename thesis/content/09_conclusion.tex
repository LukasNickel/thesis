
\chapter{Conclusion and Outlook}
\label{conclusion}

\section{Conclusion}
In the course of this work, analyis methods for CTA data have been tested.
As CTA is still in construction, the analysis is limited on the use of monte carlo data.

The necessary steps to go from low level CTA-data to a high-level analysis have been performed
using the official analysis pipeline ctapipe, that is still actively being developed.
A high-level reconstruction based on the use of random forests 
and the aict-tools package has been worked out as alternative
or complement to the existing algorithms.
Diffuse proton and gamma simulations have been used as training samples for the models, 
whereas the evaluation was done on pointlike gamma samples.

With this setup, gamma-/hadron separation works as expected and different algorithms 
for the source position reconstruction lead to roughly comparable angular
resolutions.
Due to the layout of the CTA (south) array, low energy events, on average, have lower multiplicities.
This makes is difficult to reconstruct the source position with the HillasReconstructor 
algorithm and a DISP-based approach can lead to improvements. This is especially true for
event multiplicities two or three and can thus be relevant for the early stages of operation of CTA.
At high event energies and high multiplicities, the HillasReconstructor is unrivaled.

A simplified sensitivity optimisation for both methods has been performed with
the same results regarding the performance in different energy bins.
As most parts of the analysis were the same for both methods, the results
are very similar.
An in depth comparison to the reference analysis of CTA is not feasible,
but the sensitivity seems to be not too far off.
The effective area is larger, whereas the angular separation is worse,
which points towards harder cuts - maybe also in earlier analysis steps - 
in the reference analysis.

\section{Outlook}
A lot of potential to improve the analysis is still open:

The performed preprocessing uses a rather generic cleaning, with
parameters, that were generally accepted as being 
decent, but not overly optimised in the working group. Recently, 
more relaxed parameters - especially for the LSTCam - have been used.
Furthermore, more complex cleaning methods could be inspected further.
Since the start of this thesis, two additional algorithms have been 
implemented into ctapipe:
An imcomplete three-threshold procedure used in CTA-MARS, which does not factor 
in timing information yet, and a procedure used in the FACT experiment.
The last one was added to ctapipe by me, but due to the high effort to
optimize the cleaning parameters, was not used for the analysis.
Additionaly, no cuts have been performed on the data apart from
the high level sensitivity cuts. Disregarding hard to reconstruct
events, such as events with high leakage or low intensity,
could further improve the analysis.

In the foreseeable future, new monte carlo data will be accesible as well.
This dataset will act as a common benchmark dataset in the collaboration,
which makes comparing results and optimizing parameters easier.

Future steps could also include an analysis of observed monoscopic 
LST data. Since the LST is currently the only telescope 
at LaPalma, an analysis can not rely on stereoscopic information.
The methods employed for the background separation, energy estimation and 
DISP-calculation will also work in the purely monoscopic case, although further
optimisation will probably be needed.

At last all models could be optimized in terms of the used architecture and
hyper parameters. Although random forests are generally 
robust in terms of the used hyper parameters, some gains might still be 
open. 
A recent thesis at this chair for example was based on using
extremely randomized trees for the models.
Earlier work has also shown, that an approach with two nested models
can be superiour for the energy estimation. In that case
a random forest model was trained on telescope level and
a second randm forest combined these results using the
array wide features and the combined predictions \cite{ba-lars}.
These results have been verified during the course of this thesis,
but the focus was put on the DISP based reconstruction of the
source position.