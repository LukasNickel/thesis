alles aus dem bookap aus der nextcloud

\chapter{Einleitung}


ankle? knee?

\section{astro (PARTICLE) physics}
das ist eigentlich die einleitung denke ich (sec)

bis hier hin braucht nicht mehr als 5-10 seiten sein (eher weniger)
historie, multi messenger (typen unterscheiden)


A big field in astro physics ever since Victor Hess discovered
the extraterrestrial origin of the atmosperic ionization (citation needed)
is astro particle physics. Despite falling out of favor for a few years
when terrestrial particle accelerators reached high (???) energies,
the field is very famous today. One of the reasons being that
even the best accelerators, like the LHC, got to a point where
increasing collision energies gets ever more difficult (citation needed).
Despite all the effort and numerous achievements (beispiele und belege!),
there are still many unanswered questions concerning modern
astrophysics. Some of the more prominent include:
\begin{itemize}
    \item{General relativity (-> gravitational waves)}
    \item{Dark matter (-> gravitation being weird)}
    \item{Dark energy (expansion of the universe)}
    \item{Sources of VHE cosmic rays (acceleration processes involved)}
    \item{neutrinos (basically everything, majorana?)}
\end{itemize}

Nowadays there exists a wide range of experiments trying to
solve these questions. These focus on different ways to detect
extraterestrial particle sources which is why the
the term Multi Messenger Astronomy is widely used.
Since the detection of gravitational waves in XXXX (citation needed)
there are now four seperate detection channels:

jeweils: Wofür ist das besonders gut?
\textbf{Electromagnetic / gammas}
Emitted photons can be observed either directly
from outside the atmosphere via satellites or indirectly
via ground based gamma astronomy. In the later case
IACT's are used to detect electromagnetic showers induced
by the collision of high energy photons with partticles in the atmosphere.
An example for direct observation could be the Fermi satellite (citation needed),
an example for ground based observation could be the
MAGIC-experiment (citation needed).

\textbf{Cosmic Rays}
Cosmic rays can similarly be observed both directly and indirectly.
????

\textbf{Neutrinos}
Detecting neutrinos is - in general - much harder because of
their small cross-sections. This leads to thousands (größenordnung checken) of
neutrinos passing us continiusly without us even noticing.
For this reason neutrino detectors need to be very large. One example
of such a detector is the ICECUBE detector in ??? (ctation needed).

\textbf{Gravitational waves}
Gravitational waves are the newest channel available for observation.
We can detect them using large size interferometers such as
the LiGO's three detectors (citation needed).





\section{detection of gamma rays on the ground}
-> ww in atmosphäre und so
-> messenger gamma-strahlung genauer (was untersucht man?, Vorgehen, Experimente)

% arten von quellen und wie sie beobachtet werden -> welche arten von experimenten beobachten was? wofür sind iacts besonders gut?