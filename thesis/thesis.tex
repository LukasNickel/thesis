%%%%%%%%%%%%%%%%%%%%%%%%%%%%%%%%%%%%%%%%%%%%%%%%%%%%%%%%%%%%%%%%%%%%%%%%%%%%%%%%
%%%%%%%%%%%%%%%%%%   Vorlage für eine Abschlussarbeit   %%%%%%%%%%%%%%%%%%%%%%%%
%%%%%%%%%%%%%%%%%%%%%%%%%%%%%%%%%%%%%%%%%%%%%%%%%%%%%%%%%%%%%%%%%%%%%%%%%%%%%%%%

% Erstellt von Maximilian Nöthe, <maximilian.noethe@tu-dortmund.de>
% ausgelegt für lualatex und Biblatex mit biber

% Kompilieren mit
% latexmk --lualatex --output-directory=build thesis.tex
% oder einfach mit:
% make

\documentclass[
  %oneside, % if not printed
  tucolor,       % remove for less green,
  BCOR=12mm,     % 12mm binding corrections, adjust to fit your binding % 0 if oneside
  parskip=half,  % new paragraphs start with half line vertical space
  open=any,      % chapters start on both odd and even pages
  cleardoublepage=plain,  % no header/footer on blank pages
]{tudothesis}

\usepackage{subcaption}
% Warning, if another latex run is needed
\usepackage[aux]{rerunfilecheck}

% just list chapters and sections in the toc, not subsections or smaller
\setcounter{tocdepth}{1}

%------------------------------------------------------------------------------
%------------------------------ Fonts, Unicode, Language ----------------------
%------------------------------------------------------------------------------
\usepackage{fontspec}
\defaultfontfeatures{Ligatures=TeX}  % -- becomes en-dash etc.

% german language
\usepackage[main=english, ngerman]{babel}
%\setdefaultlanguage{english}

% for english abstract and english titles in the toc
%\setotherlanguages{german}

% intelligent quotation marks, language and nesting sensitive
\usepackage[autostyle]{csquotes}

% microtypographical features, makes the text look nicer on the small scale
\usepackage{microtype}

%------------------------------------------------------------------------------
%------------------------ Math Packages and settings --------------------------
%------------------------------------------------------------------------------

\usepackage{amsmath}
\usepackage{amssymb}
\usepackage{mathtools}

% Enable Unicode-Math and follow the ISO-Standards for typesetting math
\usepackage[
  math-style=ISO,
  bold-style=ISO,
  sans-style=italic,
  nabla=upright,
  partial=upright,
]{unicode-math}
\setmathfont{Latin Modern Math}

% nice, small fracs for the text with \sfrac{}{}
\usepackage{xfrac}


%------------------------------------------------------------------------------
%---------------------------- Numbers and Units -------------------------------
%------------------------------------------------------------------------------

\usepackage[
  locale=DE,
  separate-uncertainty=true,
  per-mode=symbol-or-fraction,
  binary-units=true,
]{siunitx}
\sisetup{math-micro=\text{µ},text-micro=µ}

%------------------------------------------------------------------------------
%-------------------------------- tables  -------------------------------------
%------------------------------------------------------------------------------

\usepackage{booktabs}       % \toprule, \midrule, \bottomrule, etc

%------------------------------------------------------------------------------
%-------------------------------- graphics -------------------------------------
%------------------------------------------------------------------------------

\usepackage{graphicx}
\usepackage{grffile}
\usepackage{tabularx}

% allow figures to be placed in the running text by default:
\usepackage{scrhack}
\usepackage{float}
\floatplacement{figure}{htb}
\floatplacement{table}{htb}

% keep figures and tables in the section
\usepackage[section, below]{placeins}


%------------------------------------------------------------------------------
%---------------------- customize list environments ---------------------------
%------------------------------------------------------------------------------

\usepackage{enumitem}

%------------------------------------------------------------------------------
%------------------------------ Bibliographie ---------------------------------
%------------------------------------------------------------------------------

\usepackage[
  backend=biber,   % use modern biber backend
  autolang=hyphen, % load hyphenation rules for if language of bibentry is not
                   % german, has to be loaded with \setotherlanguages
                   % in the references.bib use langid={en} for english sources
  sorting=none,
  style=numeric,
]{biblatex}
\addbibresource{references.bib}  % the bib file to use
\DefineBibliographyStrings{german}{andothers = {{et\,al\adddot}}}  % replace u.a. with et al.

% Last packages, do not change order or insert new packages after these ones
\usepackage[pdfusetitle, unicode, linkbordercolor=tugreen]{hyperref}
\usepackage{bookmark}
\usepackage[shortcuts]{extdash}

% yaml highlighting for the appendix
\usepackage{xcolor}
\usepackage{listings}

\newcommand\YAMLcolonstyle{\color{red}\mdseries}
\newcommand\YAMLkeystyle{\color{black}\bfseries}
\newcommand\YAMLvaluestyle{\color{blue}\mdseries}

\makeatletter

% here is a macro expanding to the name of the language
% (handy if you decide to change it further down the road)
\newcommand\language@yaml{yaml}

\expandafter\expandafter\expandafter\lstdefinelanguage
\expandafter{\language@yaml}
{
  keywords={true,false,null,y,n},
  keywordstyle=\color{darkgray}\bfseries,
  basicstyle=\YAMLkeystyle,                                 % assuming a key comes first
  sensitive=false,
  comment=[l]{\#},
  morecomment=[s]{/*}{*/},
  commentstyle=\color{purple}\ttfamily,
  stringstyle=\YAMLvaluestyle\ttfamily,
  moredelim=[l][\color{orange}]{\&},
  moredelim=[l][\color{magenta}]{*},
  moredelim=**[il][\YAMLcolonstyle{:}\YAMLvaluestyle]{:},   % switch to value style at :
  morestring=[b]',
  morestring=[b]",
  literate =    {---}{{\ProcessThreeDashes}}3
                {>}{{\textcolor{red}\textgreater}}1     
                {|}{{\textcolor{red}\textbar}}1 
                {\ -\ }{{\mdseries\ -\ }}3,
}

% switch to key style at EOL
\lst@AddToHook{EveryLine}{\ifx\lst@language\language@yaml\YAMLkeystyle\fi}
\makeatother

\newcommand\ProcessThreeDashes{\llap{\color{cyan}\mdseries-{-}-}}

%------------------------------------------------------------------------------
%-------------------------    Angaben zur Arbeit   ----------------------------
%------------------------------------------------------------------------------

\author{Lukas Nickel}
\title{Stereo Reconstruction for the Early Days of CTA}
\date{27.03.2020}
\birthplace{Bielefeld}
\chair{Chair for Experimental Physics V}
\division{Faculty of Physics}
\thesisclass{Master of Science}
\submissiondate{27.03.2020}
\firstcorrector{Prof.~Dr.~Dr.~Wolfgang Rhode}
\secondcorrector{Prof.~Dr.~Bernhard Spaan}

% tu logo on top of the titlepage
\titlehead{\includegraphics[height=1.5cm]{logos/tu-logo.pdf}}

\begin{document}
\frontmatter
\maketitle

% Gutachterseite
\makecorrectorpage

% hier beginnt der Vorspann, nummeriert in römischen Zahlen
%\thispagestyle{plain}

\section*{Kurzfassung}
In den letzten Jahrzehnten hat die VHE-Astronomie (ist das ein wort?) 
viele Erkenntnisse über die Zusammensetzung des Universums gebracht.
Mitverantwortlich dafür waren die großen $\gamma$-Teleskope der 
dritten Generation. In Zukunft soll diese Forschung begleitet und erweitert werden 
durch ein Experiment der nächsten Generation.
Das Cherenkov-Teleskope-Array (CTA) soll die Sensitiviät weiter steigern und den
beobachtbaren Energiebereich erweitern.
In dieser Arbeit werden Rekonstruktionsalgorithmen für CTA vorgestellt und Studien auf 
Monte-Carlo-Daten durchgeführt mit dem Fokus der frühen Phase des Experimentes, 
in der nur wenige Teleskope errichtet sein werden.


\section*{Abstract}
\begin{english}
The abstract is a short summary of the thesis in English, together with the German summary it has to fit on this page.
\end{english}

\setcounter{tocdepth}{6}
\tableofcontents

\mainmatter
% Hier beginnt der Inhalt mit Seite 1 in arabischen Ziffern
\chapter{Introduction V0.0}
\section{astro V0.0}
kurz halten, wenig aus dem bisherigen nehmen
kurze erwähnung von multi messenger, aber neutrinos und 
grav waves juckt uns eh nicht.
gammas pointen zur quelle, hadrons nicht
wir machen gammas!
ohne viel history
\section{gamma V0.0}
\subsection{physik, die uns interessiert V0.0}
quellen, offene physik fragen in verbindung mit gamma rays.
kurz
\subsection{gamma signal, hadron background V0.0}
entstehung gammas. mit kleinen erweiterungen schon da.
kurz: hadrons gibts auch. ist für uns background, weil mans auch misst
und es viel mehr gibt. gleichzeitig für den fall nicht so interessant.
\subsection{types of experiments, vor und nachteile V0.0}
direkt vs indirekt.
satelliten, tanks und iacts -> unterschiede/vorteile/nachteile.
Wir machen indirekt mit aicts.
\subsection{iacts V0.0}
entspricht quasi 1.3 von vorher
\subsubsection{gamma shower V0.0}
passt grob
\subsubsection{hadron shower V0.0}
mini anpassungen
\subsubsection{cherenkov licht, messen am boden V0.0}
das meiste kann eigentlich weg, da in methods erklärt.

\chapter{CTA V0.0}
sensitivity relativ zum crab für alle + cta angeben 
lst,mst,sst plots vielleicht text drum wrappen?
\section{current gen V0.0}
wir haben aktuell diese experimente:
\subsection{magic V0.0}
\subsection{veritas V0.0}
\subsection{hess V0.0}
\section{wie viel besser wird cta V0.0}
vergleich zu vorher: arrays auf beiden hemisphären, verschiedene teleskope, viele teleskope
status cta aktuell.
cta besonderheiten.
auflösungsrequirements unterbringen!
\subsection{lst V0.0}
passt grob
\subsection{mst V0.0}
passt grob
\subsection{sst V0.0}
passt grob

\chapter{Methods: Relativ allgemein V0.0}
wie werden daten genommen?
what needs to be done in an analysis?
we want to simulate stuff to do supervised learning and control algorithms.
\section{Coordinate Frames}
wichtig, vor allem altaz, camera, nominal, ground
\section{Monte carlo V0.0}
ist wichtig duh
\subsection{corsika V0.0}
atmosphäre
\subsection{simtel V0.0}
telescope response
\section{ctapipe V0.0}
was ist das, warum ist das und was habe ich gemacht?
-> neu: num islands, fact cleaning
\subsection{"preprocessing" V0.0}
damit wir damit arbeiten können, muss erstmal kram gemacht werden, den
wir mal preprocessing nennen.
ctapipe ist nicht fertig, daher machen wir eigenen kram -> kai auf basis von ctapipe
liste der steps im detail.
\subsection{high level V0.0}
das sind so sachen wie energie, typ, richtung.
das kann man teilweise analytisch machen oder
per machine learning. wir nutzen nachher random forests 
\subsection{hillas reconstructor V0.0}
\section{ersatz für missing ctapipe stuff / alternative}
\subsection{disp methode, mono wie stereo V0.0}
\subsection{machine learning V0.0}
etwas angepasst das, was ich dazu schon habe.
aict tools nennen.


\chapter{Analysis: Methods spezifiziert für unser Problem! V0.0}
mit zuvor erklärten methoden zusammen fassen was getan wird
\section{preprocessing auflisten V0.0}
erhaltene features hier oder dadrunter?
\section{high level auflisten V0.0}
\subsection{gamma hadron sep V0.0}
machen wir auch per random forest 
\subsection{mono V0.0}
random forest
\subsection{stereo V0.0}
median und iterative
\subsection{cut V0.0}
ist nicht so pralle das so zu machen, aber naja
\subsection{mono modell? V0.0}
wenn zeit ist als vergleich bei multi 2&3
-> vlt machen die core und hmax schätzungen das da tatsächlich schlechter?

\chapter{Results V0.0}
\section{gamma/hadron V0.0}
\section{position mono V0.0}
\section{position stereo V0.0}
\section{position stereo cut V0.0}
\section{position stereo ohne hillas features, optimized for multi 2 ? V0.0}

\chapter{Conclusion V0.0}
% \chapter{Introduction}
An important field in astronomy and particle physics ever since Victor Hess discovered
the extraterrestrial origin of the atmospheric ionization
is astroparticle physics. Despite the rise
of terrestrial particle accelerators with ever higher beam energies,
the field is very relevant today. One of the reasons is that
even the biggest accelerators, like the LHC at CERN, got to a point where
increasing collision energies gets ever more difficult.

Even with all the previous achievements
there are still many questions about the structure of the universe, that we 
do not have an answer for yet.
Some of the prominent ones are:
\begin{itemize}
    \item{How does the space between galaxies look like? 
		Most of the known $\gamma$-ray sources are active galactic nuclei. 
		Knowing their properties allows to put constrains on
		intergalactic photon radiation and magnetic fields.}
    \item{Which particles form the dark matter and how do they interact? 
		It should be possible to detect dark matter annihilation by their emitted $\gamma$-rays.}
    \item{How exactly do relativistic outflows form in black holes?
		To this date the exact mechanism around the formation, structure and evolution of these
		jets is still unknown.}
    \item{By which processes and sources do cosmic rays get accelerated to the highest energies?
		The dtetails of the underlying processes are still unknown to the present date.}
\end{itemize}

Nowadays there exists a wide range of experiments trying to
answer these questions, not all focused on 
$\gamma$-ray physics.
In fact, they look at different particles and detection
mechanisms to detect
extraterrestrial particle sources, which is why
the term multi messenger astronomy is widely used.

\iffalse
In the following we will have a look at a brief overview 
of the emergence
of astroparticle physics and explain the key differences between 
the different observable messenger particles.
We will then focus at the 
most influential IACT experiments to motivate the construction of 
the Cherenkov Telescope Array.
\fi

\section{The Origins of Astroparticle Physics}
In 1909 Theodor Wulff measured the radiation on top of the 
Eiffel tower in Paris
\cite{horandel2013early}.
Assuming that most radiation came from the 
earth crust, the decrease was less than expected.

Several balloon flights 
of different physicists (for a contemporary overview of the field see e.g. \cite{luftelektrizitaet})
and underwater experiments of Domenico Pacini in 1910 
\cite{2011arXiv1101.3015P}
lead to the conclusion that most of the radiation must indeed be
of extraterrestrial origin, which was not immediately accepted 
in the community though \cite{bookap}.

Evidence came with the extended balloon flights of Victor Hess
\cite{Hess:1912srp}:
Not only was the intensity decrease with higher altitude not 
matching the expectations, but at a certain point at around 
\SI{2000}{\meter} the intensity 
was actually increasing again.
As this was also the case in night flights, the sun
was excluded as possible source.
A few years later, in 1929, Werner Kolhörster and 
Walther Bothe confirmed these experimental results \cite{bothe1929wesen}.

The radiation was referenced as "Höhenstrahlung" 
(see e.g. \cite{myssowsky1926versuche}) 
and "cosmic rays" (see e.g. \cite{millikan1928origin}) with 
the english term cosmic rays eventually winning out.

Experiments of Jacob Clay in 1927 
\cite{clay1927penetrating}
lead to the further insight that the main proportion of 
the radiation must carry an electric charge.

With experiments resolving around the influence of the earth magnetic 
field by 
Bruno Rossi \cite{Rossi1933},
Thomas H. Johnson \cite{PhysRev.43.834},
Luis Alvarez and Arthur H. Compton \cite{PhysRev.43.835}
in 1933,
a positive charge was conducted.

The remaining puzzles towards astroparticle and $\gamma$-ray
astronomy were solved with the discovery of antimatter in a cloud chamber
by Carl D. Anderson in 1933 \cite{PhysRev.43.491}
and the discovery of air showers by detecting coincident 
events in separated Geiger-müller tubes by 
Bruno Rossi in 1934
\cite{PhysRev.45.212}
and their later description by
the team around Pierre Auger 
\cite{RevModPhys.11.288}.

In the following years multiple particles with direct use in the 
astrophysics have been found:
The pions \cite{LATTES1947}, the muons \cite{PhysRev.52.1003}
and the neutrinos \cite{Cowan103}.

Recently the so far last discovery on this journey
was made with the measurement of the first gravitational 
waves \cite{PhysRevLett.118.221101}.

\section{Multimessenger astronomy}

Nowadays we observe extraterrestrial processes on four channels: 
\begin{enumerate}
	\item Electromagnetic radiation
	\item (Charged) cosmic rays
	\item Neutrino radiation
	\item Gravitational waves
\end{enumerate}

All of the accompanying particles have vastly different
properties and require different experimental setups.
A common thing is to observe the same sources 
on different channels to learn more about the processes that happen at 
these far distant sources. This is what is often times 
referred to as multi messenger astronomy (see e.g. \cite{RevModPhys.85.1401}, where
measurements of neutrinos and gravitational waves get combined).
Figure \ref{fig:multi_messenger} illustrates the key differences between
photons, protons and neutrinos.

Before we focus on the special field of IACTs, we will have
a brief look at the different messenger particles.

\begin{figure}
	\centering
	\captionsetup{width=0.9\linewidth}
	\includegraphics[width=0.9\textwidth]{images/astro-web-titel.jpg}
	\caption{Visualization of the behaviors of different messengers
		particles in modern astronomy.
		Photons and neutrinos travel the universe without deflection,
		because they do not carry an electric charge.
		Neutrinos interact less than photons both in the universe and in the detector,
		which leads to a high proportion passing through earth.
		Charged cosmic rays get deflected by interstellar
		magnetic fields and thus generally do not allow for a assignment to a cosmological source.
		This illustration lacks gravitational waves 
		as these have only recently been measured.
		The image is taken from the DESY-website at \cite{desy_mm_astro}.
	}
	\label{fig:multi_messenger}
\end{figure}


\subsection{Gamma Rays}
In contrast to charged cosmic rays, $\gamma$-particles point towards
their sources, allowing to search for sources of radiation.
In general $\gamma$-radiation refers to photons at all wavelength,
the term $\gamma$-rays in contrast is assigned to the very 
highest energy particles with energies above those of X-rays.
A schematic illustration of the different photon wavelengths
can be seen in figure \ref{fig:em_spectrum}.

\begin{figure}
	\centering
	\captionsetup{width=0.9\linewidth}
	\includegraphics[width=.9\textwidth]{images/em_spectrum.png}
	\caption{An overview of properties and applications of photons
		on a wide range of wavelengths.
		Radio-photons possess very low energies, gamma-rays 
		carry the very highest energies.
		Visible light lies in between, technical 
		applications exist for all the shown wavelengths.
		As indicated by the temperature bar at the bottom,
		$\gamma$-rays can not feasibly be produced by thermal radiation.
		The image is taken from \cite{wiki_em}.}
	\label{fig:em_spectrum}
\end{figure}

Creation of high-energy gamma-rays is due to a lot of 
different processes. Among these we want to give a brief 
overview of two important processes, focusing on 
photon emission from an initial pure electron distribution.
Depending on the studied case other emission factors 
might play important roles as well, such as 
the $\gamma$-production from $\pi^0$-decays in cosmic rays.

\textbf{Synchroton Radiation}

Sources of cosmic and gamma rays often times include 
high magnetic fields. Any moving charged particle will thus be
deflected perpendicular to its moving direction
due to the Lorentz-force, forcing them on a radial trajectory.
At the same time a relativistic charged particle, 
that gets accelerated radially, emits synchroton 
photons with average energy given by 
equation \ref{eq:synchroton}

\begin{equation}
	\langle E_{\gamma} \rangle \propto \frac{1}{M_P} E_P^2 B^2.
	\label{eq:synchroton}
\end{equation}

Hereby $M_P$ and $E_P$ denote the accelerated particle's 
mass and energy respectively. With the inverse mass dependency 
it is immediately evident that 
synchroton radiation plays a major role in leptonic 
emission and much less in hadronic emission.

It is important to remember that this affects the 
initial electron distribution, by reducing the electrons energy.
This is sometimes referred to as synchroton cooling.

The emitted synchroton spectrum needs to be further modified 
if the emitting region is optically thick and photons 
get absorbed by the medium.
This is always true for our cases, as the regions contains 
both magnetic fields and a high density of electrons.

\textbf{(Inverse) Compton Scattering}
In a classical particle interpretation photons and electrons 
can collide exchanging energy and altering their directions.
For the normal case of Compton scattering the electron 
is assumed to be at rest and the photon can never gain 
energy by scattering of the electron.
This can be 
seen by the increase in wavelength in equation \ref{eq:compton},
with $\lambda^{\prime}$ denoting the the scattered photons 
wavelength, $\lambda$ denoting the initial photons wavelength  
and $\Theta$ the scattering angle.

\begin{equation}
	\lambda^{\prime} - \lambda  \propto \left(1-\cos{\Theta} \right)
	\label{eq:compton}
\end{equation}


If the electron itself is moving with much higher energy
than the photon, this changes and the photon can gain substantial energy.
This is referenced as inverse Compton scattering.
In that case transforming the equations to the 
electron's frame of reference and back to the labor frame 
boosts the photons energy by a factor $\gamma$ for each transformation, 
with $\gamma$ being the Lorentz factor of the electron.

A limit to the photon energies is set by the scattering cross
section, which reduces with higher photon energies.
For low energies the cross section can be approximated by 
the Thomson cross section, for high energies
one uses the Klein-Nishina cross section.

The Synchroton Self Compton model combines the above mentioned
effects to produce a photon energy distribution from an
electron distribution, which is often times assumed to
follow a power-law spectrum initially.
The free parameters of the model can be interpreted as 
three frequencies: The minimum injection frequency $\nu_m$, 
the cooling frequency $\nu_c$ and the self-absorption frequency $\nu_a$.
Depending on the order of these parameters, different 
photon distributions can be generated.

For more general information about the generation of photon 
and cosmic ray spectra, the reader is referred to 
\cite{gaisser_engel_resconi_2016}
or
\cite{bookap}.

For a detailed analysis of the influence of the parameters of
the Synchroton Self Compton model, 
the lecture of \cite{10.1093/mnras/stt1461} is advised.


Emitted $\gamma$-rays can be observed either directly
from outside the atmosphere via satellites or indirectly
via ground based gamma astronomy. In the later case
IACTs are used to detect electromagnetic showers induced
by the collision of high energy photons with particles in the atmosphere.
% An example for direct observation could be the Fermi 
% Gamma-Ray Space Telescope \cite{Atwood_2009},
% an example for ground based observation could be the
% MAGIC-experiment \cite{LORENZ2004339}.

\subsection{Cosmic rays als background erwähnen}


\iffalse
\subsection{Neutrinos}
Even less affected by interactions on the way 
from the source to the earth are neutrinos ($\nu$).
Due to their small interaction cross sections and no electric charge, they 
suffer very little from absorption or deflection.
For very much the same reasons detecting neutrinos is much harder
than detecting photons or charged particles.

The small cross section requires to build huge detectors, 
the ICECUBE having a detector volume of \SI{1}{\kilo\meter^3}
\cite{Abbasi:2008aa}.
An illustration of the ICECUBE Neutrino Observatory
is shown in figure \ref{fig:icecube}.
\begin{figure}
	\centering
	\includegraphics[width=0.8\textwidth]{images/icecube.png}
	\caption{icecube illustration, cite that \cite{Abbasi:2008aa}}
	\label{fig:icecube}
\endrvation.
They are a direct result from general relativity and occur
at certain constellations of very high masses, such as 
merging black holes.

An illustration of such an event is shown in figure \ref{fig:gravi_waves}. 
\begin{figure}
	\centering
	\includegraphics[width=0.8\textwidth]{images/cr_spectrum.png}
	\caption{images}
	\label{fig:gravi_waves}
\end{figure}


We can detect them using large size interferometers such as
the LiGO's three detectors (citation needed).
\fi



\section{Detection of Gamma Rays}

\subsection{Andere Methoiden, warum iacts?}
\subsection{Kram drunter als IACT}
The primary particles of gamma or cosmic rays cannot be 
observed with IACTs directly. Instead one can measure the secondary particles
that emerge from the particles interaction with matter.

If the primary particle energy is high enough, the resulting 
secondary
particles can interact with the atmosphere themself, thus starting a 
cascade of secondary particles.

Depending on whether the primary particle is 
a photon/electron/positron or a heavier particle such as a proton 
or iron core, the interactions vary.

This leads to a separation of electromagnetic and hadronic showers.
If the experiment is primarly looking for 
gamma rays, e.g. if measuring a known source like the Crab Nebula, 
the hadronic showers act as dominant background.
As hadronic showers get observed much more frequently, 
the identification of the primary particle type is a very important 
task, often times referred to as gamma-/hadron-separation.

To understand how these showers differ, we will have a very brief look
at the relevant interactions at the primary particle energies
we want to observe.

\subsection{Electromagnetic Showers}
Electromagnetic showers consist mainly of three types of particles:
\begin{enumerate}
	\item{Photons $\gamma$}
	\item{Electrons $e^-$}
	\item{Positrons $e^+$}
\end{enumerate}

The main interaction for high energy photons is pair 
production, generating an $e^+/e^--$pair where the energy of 
the secondary particles equals the photon energy.
On the other hand high energy electrons (and positrons) lose 
most of their energy by radiation, leading to a photon with 
an energy close to the electron energy.
Only at lower particle energies other interaction forms show their impact,
with particle scattering and ionization 
leading to more continuous energy losses.

These assumptions lead to the most basic model of an 
electromagnetic shower, proposed by Bhabha and Heitler in 1937
\cite{doi:10.1098/rspa.1937.0082}.
It starts with a high energy primary photon before its interaction in the atmosphere 
(The original paper starts from a high energy electron interacting in lead actually.
As will be evident this does not concern our qualitative look at the model.) and continues 
the calculation in discrete epochs.
The photon produces a pair of $e^-$ and $e^+$ in the first epoch.
Because of the high energies at play, the direction of these secondary 
particles does not deviate significantly from the photons direction, 
making the problem essentially one-dimensional.
The $e^+/e^-$ continue on to radiate a photon each and the cycle continues.
Each step doubles the number of particles in the shower with each particle 
on average getting half the energy of its parent particle.
These processes continue until the energy of the $e^+/e^-$ becomes low enough for
continuous ionization processes to become relevant.
At this point the particle is considered to be stopped and the shower
does not evolve further.

Today Monte Carlo calculations get used to simulate the properties 
of particle showers in the atmosphere.
The most common software to model the atmospheric interactions is
CORSIKA \cite{Engel:2018akg}.

Figure \ref{fig:gamma_shower} illustrates the Bhabha-Heitler model (left)
and a \SI{100}{\giga\electronvolt} gamma shower, simulated with CORSIKA (right).

\begin{figure}
	\centering
	\captionsetup{width=0.9\linewidth}
	\begin{subfigure}{.7\textwidth}
  		\centering
  		\includegraphics[width=\linewidth]{images/em_shower_illustration.png}
	\end{subfigure}%
	\begin{subfigure}{.2\textwidth}
 		\centering
		\includegraphics[width=\linewidth]{images/corsika_100gev_photon.png}
	\end{subfigure}
	\caption{Schematic illustration and simulation of a $\gamma$ air shower. \\
		Left: A schematic illustration of the first epochs of the 
		Bethe-Heitler shower model with equal radiation lengths for
		bremsstrahlung and pair production.
		The image is taken from an inaugural thesis 
		by Stefan Funk \cite{funk_doctor}. \\
		Right: A \SI{100}{\giga\electronvolt} gamma-shower as xz-projection, simulated with CORSIKA.
		The shower is relatively contained with only little extent perpendicular 
		to the shower direction (z-axis). The image is taken from 
		the CORSIKA-website \cite{corsika_showers}}
	\label{fig:gamma_shower}
\end{figure}

\subsection{Hadronic Showers}
Hadronic shower include all the interactions known from 
electromagnetic showers, but add nuclear interactions on top.
These lead to non-negligible additional energy losses 
and the creation of secondary hadronic particles.

Approximations are more difficult to do and monte carlo simulations 
become the only way to reasonably calculate shower behavior.

At the end of the shower a relevant portion of the particles have decayed into the 
lightest hadronic particles, pions ($\pi^0, \pi^+, \pi^-$), of which the neutral pions 
rapidly decay into photons.
This means that a part of the hadronic shower
eventually becomes an electromagnetic subshower.

Figures \ref{fig:proton_shower}
shows some of the particles generated in a hadronic shower (left)
and a \SI{100}{\giga\electronvolt} proton shower, simulated with CORSIKA (right).

\begin{figure}
	\centering
	\captionsetup{width=0.9\linewidth}
	\begin{subfigure}{.7\textwidth}
  		\centering
  		\includegraphics[width=\linewidth]{images/hadron_shower_illustration.png}
	\end{subfigure}%
	\begin{subfigure}{.2\textwidth}
 		\centering
		\includegraphics[width=\linewidth]{images/corsika_100gev_proton.png}
	\end{subfigure}
	\caption{Schematic illustration and simulation of a proton air shower.\\
		Left: A purely qualitative,
		schematic illustration of the generation of a hadronic shower.
		It shows how the primary hadron generates different subshowers
		with vastly different particles.
		The image is taken from an inaugural thesis 
		by Stefan Funk \cite{funk_doctor}.\\
		Right: A \SI{100}{\giga\electronvolt} proton-shower as xz-projection,
		simulated with CORSIKA.
		The shower is less contained than the gamma shower of equal energy in 
		figure \ref{fig:gamma_shower}.
		Different colors indicate different particle types.
		The image is taken from 
		the CORSIKA-website \cite{corsika_showers}.}
	\label{fig:proton_shower}
\end{figure}

\subsection{Measuring Events}
\label{sec:measuring}

The way IACTs detect particle showers in the atmosphere is by their emittance 
of Cherenkov light. Cherenkov light gets emitted whenever a 
charged particle moves through a dielectric medium with a velocity 
exceeding the local speed of light, which happens in air for the
highly relativistic particles we are lokking at.

Cherenkov photons get collected by the mirror(s) of a telescope
and projected onto a camera system mounted above the mirror.
With this setup IACTs usually reach a field of view of a few degree.
Figure \ref{fig:iact_mirror_camera} illustrates this setup.

\begin{figure}
	\centering
	\captionsetup{width=0.9\linewidth}
	\includegraphics[width=0.9\textwidth]{images/cta47.png}
	\caption{A schematic illustration of the working principles of 
	an IACT experiment:
	A $\gamma$-ray produces an air shower in the atmosphere
	that points roughly towards the telescopes.
	Cherenkov light from the air shower 
	hits the mirrors and gets focused into a camera mounted on top.
	An illustration of the resulting image after integrating 
	\SI{10}{\nano\second} of the pixel measurements
	can be seen in the left bottom corner.
	The image is taken from the official CTA-website \cite{cta_web}.}
	\label{fig:iact_mirror_camera}
\end{figure}


Upon detection of a shower and performing the low-level analysis,
the usual tasks of a high-level analysis include reconstructing 
three key properties of the primary particle:
\begin{enumerate}
	\item{Shower type (e.g. gamma or hadronic)}
	\item{Primary particle energy}
	\item{Shower direction/source position}
\end{enumerate}

In most cases the reconstruction can be improved heavily by cleaning the image first.
Figure \ref{fig:shower_cleaning} shows a simulated $\gamma$-induced shower image
in the camera of the Large Sized Telescope (LST, see section \ref{sec:lst} for more details of the telescope)
before and after cleaning.

\begin{figure}
	\centering
	\captionsetup{width=0.9\linewidth}
	\begin{subfigure}{.45\textwidth}
  		\centering
  		\includegraphics[width=\linewidth]{Plots/hillas_raw.pdf}
	\end{subfigure}%
	\begin{subfigure}{.45\textwidth}
 		\centering
		\includegraphics[width=\linewidth]{Plots/hillas_cleaned.pdf}
	\end{subfigure}
	\caption{An idealized gamma-shower in the camera of a LST
	    before and after cleaning.
		The pixel colors show the intensity as integrated 
		waveform in each pixel. 
		Brighter color indicates higher intensity.\\
		Left: The signal directly after
		the waveform integration. Non-signal pixels are noisy. 
		Right: Image after the cleaning-algorithm has been 
		applied. The noise in the non-signal pixels is gone,
		improving the analysis.}
	\label{fig:shower_cleaning}
\end{figure}

The analysis then usually involves describing the 
shower image as an ellipse and calculating the so called hillas parameters,
allowing the description of the shower image with only a handful of parameters.
The historical approach can be read up in 
a famous paper of A.M. Hillas \cite{hillas_params}.

An illustration with the hillas ellipse on top of the shower image 
can be seen in figure \ref{fig:hillas_params}.
This also includes an estimate of the true source position with the DISP-method
which we will use and explain at later stages.

\begin{figure}
	\centering	
	\captionsetup{width=0.9\linewidth}
	\includegraphics[width=.6\textwidth]{Plots/hillas_complete.pdf}
	\caption{The same cleaned shower-image as in figure \ref{fig:shower_cleaning}
	with the hillas-parameters calculated and marked in the camera frame.
	The parameters $\phi$, $\psi$, $x$ and $y$ describe the 
	orientation and position of the shower ellipse in the camera frame.
	The shower axis constrains the possible source position. 
	Applying the DISP-method two possible points remain.}
	\label{fig:hillas_params}
\end{figure}

With the hillas parameters calculated, the reconstruction of the 
primary particles properties can be done:

The primary \textbf{particle energy} is mostly described by the contained light in the image combined with 
an estimate of how much of the light missed the camera. 

The \textbf{particle type} can be reconstructed 
by looking at the image shape.
As we saw earlier, $\gamma$ and hadronic showers have different properties, 
resulting in different images. The assumption of a single, elliptic signal
area only works reasonably well for $\gamma$-showers.
A representation of different shower types can be seen in figure \ref{fig:compare_showers}.

\begin{figure}
	\centering
	\captionsetup{width=0.9\linewidth}
	\includegraphics[width=.9\textwidth]{images/shower_types.png}
	\caption{A comparison of different shower types as seen 
	by the Whipple telescope.
	Hadronic showers are more separated than gamma showers and tend to
	form multiple clusters in the camera.
	Muons often times produce ring-like images in the camera (WHY?).
	The image is taken from an introduction lecture by Albrecht Karle \cite{icecube_showers}.}
	\label{fig:compare_showers}
\end{figure}

For the reconstruction of the \textbf{source position}
one possible way involves using the DISP-method in some way.

As basic estimation one can predict the shower origin by 
transforming the center of gravity (cog) of the image onto the sky plane.
The DISP-method assumes that the 
center of gravity is displaced relative to the
real source position depending on the angle the photon arrived at the telescope.
With higher angles the shower ellipse gets more eccentric as well.
The ellipsicity of the ellipse is thus a measure for the displacement of the true 
source position.
If one assumes the true source position to be on the main shower axis of the ellipse,
finding this position simplifies to finding a single point on the main shower axis.

Monoscopic experiments, that make use of the DISP method, need to resolve the head-tail-ambiguity:
Knowing the distance from the cog leaves two possible points, one on either side 
of the ellipse.
Stereoscopic experiments can resolve this ambiguity by combining the images from 
multiple telescopes, as can be seen in figure \ref{fig:stereo_shower}.

\begin{figure}
	\centering
	\captionsetup{width=0.9\linewidth}
	\hspace*{0.1\textwidth}\includegraphics[width=0.9\textwidth]{images/stereo_shower.png}
	\caption{An illustration of an air shower captured with multiple telescopes.
		Each telescope captures an elliptic image.
		The image axes get intersected to reconstruct the arrival direction
		of the primary particle.
	    The image is taken from \cite{2015arXiv151005675H}.}
	\label{fig:stereo_shower}
\end{figure}



% \chapter{Cherenkov Astronomy and CTA}
\label{cta}


\section{A Brief History of Ground Based Atmospheric Cherenkov Astronomy}
In the time between the discovery of cosmic radiation by Victor Hess
and today there have been various different attempts to investigate its origins.
Experiments have been conducted in various forms like Cherenkov telescopes, scintillator 
arrays and satellites.


\subsection{The Current Generation}

Towards the end of the 1990s several third generation experiments were
proposed:
H.E.S.S and MAGIC, deriving from parts of the HEGRA collaboration, 
VERITAS from Whipple and the no longer operating CANGAROO from Adelaide and 
several japanese universities \cite{HILLAS201319}. All of these
were designed as stereoscopic imaging telescopes building on the progress made during the 
second generation with two experiments located on each the north and the south hemisphere.

\subsubsection{Major Atmospheric Gamma Imaging Telescopes (MAGIC)}
MAGIC is a experiment nowadays operating as a stereo setup of two telescopes.
Both telescopes are located at La Palma and have a \SI{17}{\meter} diameter mirror \cite{ALEKSIC201676}.

In the first phase MAGIC consisted of a single telescope, for differentiation usually
referred to as MAGIC-I, making MAGIC a second generation experiment. This first phase started 
operation in 2004 and had MAGIC-I be the largest IACT of its time.

The addition of MAGIC-2, the second mostly identical telescope, in 2012 marked the 
start of the second phase of the experiment and the transition to a third 
generation experiment \cite{2009arXiv0907.1211C}.
In the second phase MAGIC operates in the energy range from \SI{30}{\giga\eV}
up to \SI{100}{\TeV} \cite{magic_website}.

The DISP-method for the reconstruction of the event arrival direction 
was significantly improved by including timing information and showed better 
results for stereoscopy than the simple crossing of the main shower axis \cite{ALEKSIC2012435}.

A specialty of the MAGIC experiment lies in its ability to 
perform fast skews and thus react to gamma ray bursts after they have been observed 
by satellite experiments \cite{2003ICRC....5.2943B}.
This enabled the recent observation of the very first \si{\TeV}
gamma ray burst observed with IACTs \cite{collaboration2019teraelectronvolt}.


\subsubsection{Very Energetic Radiation Imaging Telescope Array System (VERITAS)}
VERITAS	was initially planned as a seven-telescope array, arranged in a diamond shape
\cite{WEEKES2002221}.

Eventually the collaboration settled 
with four telescopes. 
Each telescope covers a 3.5° FoV with a \SI{12}{\meter} diameter mirror and 
a 499 pixel PMT camera.

A relocation of telescope 1 in 2009 meant that the 
experiment made better use of the given area and its four telescopes,
improving the sensitivity by up to 30\% \cite{2009arXiv0912.3841P}.
The old and new layout can be seen in figure \ref{fig:veritas_relocation}.

The VERITAS collaboration mentions a lower threshold of \SI{100}{\giga\electronvolt}
with the four telescope setup.

\begin{figure}
	\center
	\captionsetup{width=0.9\linewidth}
	\includegraphics[width=.9\textwidth]{images/veritas_relocation.png}
	\caption{The VERITAS array layout after relocation of the first 
		telescope. The distances between the telescopes are 
		highlighted for the old(blue) and the new(red) array.
		In the old setup the telescopes T1, T2 and T3 were too closely
		located, making their measurements redundant.
	 	The image is taken from an official paper, that investigates
		the improved performance gained by relocating the telescope \cite{2009arXiv0912.3841P}.}
	\label{fig:veritas_relocation}
\end{figure}

Like MAGIC, VERITAS makes use of the DISP-method, especially at large zenith angles 
\cite{2015ICRC...34..771P}.

\subsubsection{High Energy Stereoscopic System (HESS)}

The HESS experiment consists of five telescopes and 
is - in contrast to MAGIC and VERITAS - operating in the southern 
Hemisphere, in Namibia.

A similar distinction in phase I and phase II as with MAGIC can be taken with 
HESS phase I consisting of four \SI{13}{\meter} diameter telescopes,
arranged at the edge of a square of \SI{120}{\meter} long sides \cite{HINTON2004331}.
These operated from 2004 to 2012 when the experiment went into phase II.

Phase II brought the larger, fifth telescope with the aim to lower the energy threshold
further. This $\SI{24}{\meter} \times \SI{32}{\meter}$-mirror telescope 
was placed in the middle of the other telescopes.
HESS operates at a similar energy range as MAGIC, stating a lower threshold as low as 
\SI{20}{\giga\electronvolt} \cite{vincent2005hess}.


\section{Next to come: The Cherenkov Telescope Array (CTA)}
\label{sec:cta}

The Cherenkov Telescope Array aims to be a next generation IACT experiment.
With two sites of operation, one for each hemisphere, and a number of different 
telescopes proposed, CTA is going to expand on the findings of the third 
generation experiments.

Like HESS in phase II, the CTA arrays are going to consist of different sized telescopes, namely
the Large Sized Telescope (LST, \SI{23}{\meter}), 
the Medium Sized Telescope (MST, \SI{12}{\meter}) 
and the Small Sized Telescope (SST, \SI{4.3}{\meter}).

Extensive Monte Carlo simulations have been performed to find optimal array arrangements
\cite{BERNLOHR2013171}.

The currently planned layouts at LaPalma and in Chile are shown in 
\ref{fig:cta_layout}

\begin{figure}
	\center
	\captionsetup{width=0.9\linewidth}
	\includegraphics[width=0.9\textwidth]{images/cta_layout.png}
	\caption{The planned layouts for CTA. The dots mark the telescope types.
	Left: The array on the northern hemisphere. It will be build at LaPalma
	and feature only LSTs and MSTs.
	Right: The array at the southern hemisphere.
	It will consist of more telescopes and thus 
	work at a bigger energy range.
	\cite{cta_web}}
	\label{fig:cta_layout}
\end{figure}


\subsection{LST}
\label{sec:lst}

The Large-Sized Telescope is going to be biggest telescope of CTA
with a mirror diameter of \SI{23}{\meter}.
It will provide the best sensitivity in the energy range from 
\SI{20}{\giga\electronvolt} to \SI{150}{\giga\electronvolt} with a field of view of \SI{4.3}{\degree}.
The camera of the LST, the LSTCamera, has \num{1855} channels 
with \num{265} photomultiplier tubes \cite{cta_web}.
The readout electronic is based on the Domino Ring Sampler 
Version 4 chip, which is also used by the MAGIC experiment
\cite{Kubo:2013pwa}. 

Since the LST is looking for the lowest energy $\gamma$-rays, it needs
very large mirror areas. At the same time the effective detector area does 
not need to be as high as for higher energy events.
For this reason only 4 telescopes are planned per array.

The first LST has been inaugurated at the 10 October 2018 in La Palma \cite{lst_debut}.

\begin{figure}
	\center
	\captionsetup{width=0.9\linewidth}
	\includegraphics[width=.4\textwidth]{images/LST.png}
	\caption{
		An illustration of the Large Sized Telescope (LST) with its
		\SI{23}{\meter} diameter parabolic mirror.
		In the lower left corner a human is included for scale.
		The image can be found at the official CTA-website \cite{cta_web}.}
	\label{fig:lst}
\end{figure}

\subsection{MST}

The Medium-Sized Telescopes (MST) are primarily going to look at the 
energy range from \SI{150}{\giga\electronvolt} to \SI{5}{\tera\electronvolt}.
A total of 15 telescopes on the north site and 25 telescopes at the south site 
are going to be the backbones of CTA.
Two camera designs are being tested for the MST:
The 1764 pixel FlashCam and the 1855 pixel NectarCam \cite{cta_web}.

\begin{figure}
	\center
	\captionsetup{width=0.9\linewidth}
	\includegraphics[width=.4\textwidth]{images/MST-1.png}
	\caption{An illustration of the Mid Sized Telescope (MST) with its
	\SI{11.5}{\meter} diameter mirror.
	The mirror design is based on the Davies-Cotton design.
	The image can be found at the official CTA-website \cite{cta_web}.}
	\label{fig:mst}
\end{figure}



\subsection{SST}
The Small-Sized Telescopes (SST) will provide the sensitivity for CTA at the 
highest energies upwards from \SI{5}{\tera\electronvolt}.
An upper limit for the sensitivity is expected to be around \SI{300}{\tera\electronvolt}.

The design for the SST includes two mirrors, a \SI{4.3}{\meter} and a \SI{1.8}{\meter}
mirror, before the light hits the camera.

In contrast to the LST and MST, the SST's camera includes silicon photo-multipliers
and a total of 2368 pixels. With this the SST is going to cover a field of view 
of \SI{10.5}{\degree}.

The north array is not going to include any SSTs, the 
south array on the other hand will contain a total of 70 telescopes over
several square kilometers \cite{cta_web}.

\begin{figure}
	\center
	\captionsetup{width=0.9\linewidth}
	\includegraphics[width=.9\textwidth]{images/sst.jpg}
	\caption{An illustration of the Small Sized Telescope (SST) with its
	two mirrors.
	The dual-mirror Schwarzschild-Couder setup includes mirrors of
	\SI{4.3}{\meter} and \SI{1.8}{\meter} diameter.
	The image can be found at the official CTA-website \cite{cta_web}.}
	\label{fig:sst}
\end{figure}

% \chapter{Methods for IACT Analyses}

\section{Coordinate Frames}
In the following the coordinate frames, that are used in this analysis,
are defined as per the ctapipe conventions \cite{karl_kosack_2019_3372211}.

\subsection{AltAz/Horizon Frame?}
A spherical coordinate system with the two angles azimuth and altitude.
This is used to describe the position of stars and shower origins.
In ctapipe, Azimuth is oriented east of north and altitude between the horizon 
and the zenith.

\subsection{CameraFrame}
A 2D cartesian frame to describe the shower in the focal plane.
The telescope orientation is defined as 
starting at the horizon, then pointed to the 
magnetic north in azimuth and up to zenith.
The two coordinates are defined as they are in CORSIKA:
$X$ points north and $y$ points west, with the telescope


\subsection{NominalFrame}
A common, spherical frame for all telescopes to perform 2D-reconstruction,
such as DISP-methods or intersection of Hillas-axes.
This is also necessary if telescopes are in divergent pointing mode.


% \subsection{GroundFrame}
% A cartesian frame with the origin in the array center. The positions of the
% telescopes is saved in the GroundFrame.

% \subsection{TiltedGroundFrame}
% For impact reconstruction, we dont need that?

\section{Monte Carlo Simulations}
Monte carlo simulations are a crucial part to 
understand the behaviour of an experiment. 
Simulating the underlying physic and the response of the experiment
allows the testing of new algorithms aswell as improving the general understanding
of the physical processes at work. 

For CTA, monte carlo simulations are done using a combination of the
programs CORSIKA and sim\_telarray.
While CORSIKA handels the formation and propagation of air showers,
the detector response can be simulated using the software
sim\_telarray \cite{BERNLOHR2008149}.

\subsection{Air Shower Simulation with CORSIKA}
CORSIKA \cite{heck1998corsika} is the standard software to simulate different types of
air showers in the field of astroparticle physics.
It allows to choose primary particles at will and
propagate them through the atmosphere, generating secondary particles, until
eventually the ground is reached.
Simulation options include, amongst other:
\begin{itemize}
  \item Amount of data to simulate
  \item Several random seeds
  \item Primary particle type
  \item Range and slope of the energy spectrum
  \item Alt/az range
  \item Models and parameters for the electromagnetic and hadronic interactions
  \item Models and parameters for the cherenkov light generation
  \item Atmospheric properties, such as the magnetic field
  \item Position and size of the detector grid
\end{itemize}

\subsection{Detector Simulation with sim\_telarray}
sim\_telarray is a software developed for the HEGRA IACT experiment,
but has since been developed to support arbitrary experiment setups \cite{BERNLOHR2008149}.
% It is able to simulate the full detector response and can be used to either
% perform complete simulations or use short-cuts in order to
% improve on the computational efficiency.

The main steps of the anaysis concern the optical raytracing of the cherenkov photons
and the simulation of the readout electronics and triggers.

The resulting data is meant to resemble the telescope raw data as closely as possible,
be it digitized pulse shapes or integrated signal sums.

\section{CTA's Processing Pipeline: ctapipe}
The main processing pipeline for low-level  CTA data is going to be ctapipe,
an open-source project hosted on github cite{ctapipe}.
It is currently in development and not all the planned 
features are yet available.
At the time of writing, the latest stable version is 0.7
\cite{karl_kosack_2019_3372211}.
Besides handling CTA-data, ctapipe also includes code to work with
other telescope geometries such as MAGIC.

\subsection{Telescope Level}
After recording an event, the data gets analyzed on telescope level.
This means that no information gets shared between telescopes.
The initial step consists of calling the
CameraCalibrator class. This includes an optional
data volume reduction (?) and an integration of the recorded waveforms.
In this step the information gets reduced to two values per pixel:
Charge and pulse time.

The resulting 2D-images gets cleaned to select the signal pixels,
that get used for further analysis.
An illustration of the cleaning step 
for a simulated shower in the LSTCam is given in figure \ref{fig:shower_cleaning}.
The default tailcuts cleaning algorithm works with two thresholds on the 
pixel charge. All pixels above the upper threshold with a set amount of neighbouring 
pixels above that threshold get selected, aswell as 
all pixels above the lower threshold with the same set amount of neighbouring pixels
above the upper threshold.
During the course of this thesis, I added a second algorithm, that is used by the FACT-collaboration 
and additionaly makes use of the pulse time information.
It works in a similar way, but with additional steps removing pixels, that arrived
at distant times.

\begin{figure}
	\centering
	\captionsetup{width=0.9\linewidth}
	\begin{subfigure}{.42\textwidth}
  		\centering
  		\includegraphics[width=\linewidth]{Plots/hillas_raw.pdf}
	\end{subfigure}%
	\begin{subfigure}{.48\textwidth}
 		\centering
		\includegraphics[width=\linewidth]{Plots/hillas_cleaned.pdf}
	\end{subfigure}
	\caption{
		Left: The signal directly after
		the waveform integration. The non-signal pixels contain a lot of noise. \\
		Right: Image after the cleaning-algorithm has been 
		applied. The highlighted pixels are used for further analysis steps.}
	\label{fig:shower_cleaning}
\end{figure}

Several parameters can be calculated on the charge values, the most prominent ones are the 
hillas parameters, illustrated in figure \ref{fig:hillas_params}.

\begin{figure}
	\centering	
	\captionsetup{width=0.9\linewidth}
	\includegraphics[width=.6\textwidth]{Plots/hillas_cleaned_params.pdf}
	\caption{The same cleaned shower-image as in figure \ref{fig:shower_cleaning}
	with the hillas-parameters calculated and marked in the camera frame.
	The parameters $\phi$, $\psi$, $x$ and $y$ describe the 
	orientation and position of the shower ellipse in the camera frame.
  The blue line marks the main shower axis.}
	\label{fig:hillas_params}
\end{figure}

Other parameters include variations of the leakage and concentration of the shower image.
Leakage describes how much of the signal is located in the outermost pixels of the camera, 
concentration describes how contained the image is by comparing the light in 
few selected pixels with the sum of all pixels.

In addition, I added an algorithm, that counts the seperated clusters in the image,
the sum of which is referred to as number of islands.

A class of parameters, that is evaluated on the pulse times, are the timing
parameters. These describe the temporal evolution of the image by fitting
a 1D-polynom on the pulse times. The parameters of the fit and the deviation
between observed and estimated pulse times are saved as the timing parameters. 


\subsection{Array Level}
From the image descriptions, high level features can be calculated.
These include the primary particle type and energy aswell as 
the shower origin and impact point.

Since CTA is going to be a stereoscopic experiment,
the information of multiple telescopes can be combined to
reconstruct propoerties of the shower and its primary particle.

An algorithm, that makes use of the stereoscopic array planned for CTA,
is the HillasReconstructor, which works based on the hillas parametrisation
of the images.

For each triggered telescope, a 2D-plane is drawn into the spatial 3D-space based on the main shower 
axis and the telescope orientation. These planes intersect and 
the weighted average of all intersections yields the 
main interaction point of the observed shower and the source position.
Intersecting the main shower axes on the ground frame leads to 
an estimation of the impact point of the shower (IS THAT HOW IT WORKS?).
For a general illustration see figure \ref{fig:hillas_reconstructor}.

Right now this is the default reconstruction algorithm inside ctapipe.
The current implementation does not provide an 
estimation for the uncertainty of the reconstructed parameters.

\begin{figure}
	\centering
	\includegraphics[width=0.6\textwidth]{images/hillas_reco.png}
	\caption{Illustration of a stereoscopic reconstruction of the source position.
    From each telescope a plane is drawn, the direct intersection 
    on the ground leads to the impact point (red circle).
    The image is taken from a habilitation thesis by 
    Mathieu de Naurois \cite{hillas_reco}.
    Not included is the reconstruction of the main interaction point
    and source position, which would be obtained by averaging the plane
    intersections.}
	\label{fig:hillas_reconstructor}
\end{figure}

Since this method inherently requires a stereoscopic experiment
and multiple triggered telescopes, it will not work for a single telescope.


% HillasIntersection auch noch erklären als Motivation für DISP?
% Aber dann müsste man eigentlic dagegen benchmarken...


\section{Additional Approaches}
With all the image features calculated and lots of monte carlo
data available, machine learning algorithms can perform
very well in these kind of analyses.
Two use cases involve the DISP-method
and the gamma-/hadron-separation.

\subsection{The DISP-Method}
For the reconstruction of the source position
one possible way involves using the DISP-method in some way.
Its advantage over the HillasReconstructor lies in the ability to
perform purely monoscopic reconstructions.

As basic estimation one can predict the shower origin by 
transforming the center of gravity (cog) of the image onto the sky plane.
The DISP-method assumes that the 
center of gravity is displaced relative to the
real source position depending on the angle the photon arrived at the telescope.
If one assumes the true source position to be on the main shower axis of the ellipse,
finding this position simplifies to finding a single point on the main shower axis, that 
can then be transformed to the horizon frame.
The calculation of the DISP is based on either lookup tables or machine learning,
both based on the use of monte carlo data.

Monoscopic experiments, that make use of the DISP method, need to resolve the head-tail-ambiguity:
The DISP calculation itself only grants the distance between
cog and source position, leaving two valid points.
One way to solve this problem is to treat it as classification task with two
classes $\pm1$, denoting the two directions to move from the cog.
Similar models as for the DISP can then be calculated using monte carlo data.

Stereoscopic experiments can resolve this ambiguity by combining the images from 
multiple telescopes, as can be seen in figure \ref{fig:stereo_shower}.

\begin{figure}
	\centering
	\captionsetup{width=0.9\linewidth}
	\hspace*{0.1\textwidth}\includegraphics[width=0.9\textwidth]{images/stereo_shower.png}
	\caption{An illustration of an air shower captured with multiple telescopes.
		Each telescope captures an elliptic image.
		The image axes get intersected to reconstruct the arrival direction
		of the primary particle.
	    The image is taken from \cite{2015arXiv151005675H}.}
	\label{fig:stereo_shower}
\end{figure}

\subsection{Classification}
One important task, that ctapipe does not handle (yet?), is the classification of
the primary particle. If looking at a gamma source this simplifies
to binary classification with the two classes signal/gamma and background/hadron.

This can be done on telescope or array level, using all the features from the previous
steps. Models are trained on the monte carlo data.

\subsection{Machine Learning}
The mentioned approaches rely on the training of machine learning models on
monte carlo data.
With full information available on the simulated data, supervised
algorithms such as random forests become feasible.

Methods based on Decision Trees work by recursively partitioning
the parameter space until the remaining samples behave similar enough.
Tree-based methods can be applied to both classification and regression tasks.

Forest methods combine multiple tree predictors to get more stable
predictions.

\subsubsection{Supervised learning}
In the task of supervised machine learning a model is trained on a
dataset with full information available.
Training in this context means to optimize the parameters of a 
model to get a better prediction on the training data.
This data will come from Monte Carlo simulations in our case, but
could also be e.g. historical or hand labeled datasets in other contexts.
The trained model can then later be used to estimate features on a dataset, which
lacks the needed information.

We define a dataset as having a number of samples with a fixed number of
features each. In the following we split
the features of our dataset into a set of input variables $X$ and
a set of output variables $y$.

The naming convention for
these sets follows the one of scikit-learn
\cite{scikit-learn}, \cite{sklearn_api}, a python package for
machine learning algorithm, that will be used for the
machine learning models in this analysis.
Other terminologies for the two feature sets include
predictors or independent variables for the input, and
responses or dependent variables for the output.

\subsubsection{Classification}
In (supervised) classification tasks, the task is to predict of which of some
predefined classes the given sample is a member. The possible solutions for $y$
are from a discrete set of values in
contrast to a regression problem with a continuous solution space.
A model that performs classification on data is referred to as a
classifier.

The simplest and most popular case of classification problems
is \textit{binary classification} \cite{sokolova2009systematic}.
In this case only two distinct
classes exist, which fortunately is all thats needed for
the basic background rejection used later.
% A common example for a classification problem is an Email spam filter,
% where mails get categorized in at least two categories based
% on their content and meta data \cite{DBLP:journals/corr/cs-CL-0006013}.

For binary classification a set of measures
to define the quality of the prediction can be defined, starting with the confusion matrix
shown in table \ref{tab:confusion},
with $pos$ referring to the true label of the positive (i.e. signal)
and $neg$ referring to the label of the negative (i.e. background) class.

\begin{table}
    \caption{Definition of the confusion matrix for binary classification.
    The main diagonal includes the correct predictions, wrong predictions are on the off diagonal.}
    \begin{center}
        \begin{tabular}{ l| l l}
            %\hline
            {} & Predicted as $pos$ & Predicted as $neg$ \\
            \hline
            $pos$ & true positive ($tp$) & false negative ($fn$) \\
            %\hline
            $neg$ & false positive ($fp$) & true negative ($tn$) \\
            %\hline
        \end{tabular}
    \end{center}
    \label{tab:confusion}
\end{table}

An ideal classification would result in
\begin{equation*}
  fp = fn = 0.
\end{equation*}

As this can usually not be achievd, different measures
can be used to examine the classifiers performance
depending on the goal of the analysis.
Some of the more common ones are listed in table \ref{tab:class_metrics}.
(Nur die genutzten? Oder die erwähnen?)

\begin{table}
    \caption{Popular metrics for binary classification tasks.}
    \begin{center}
        %\caption{
         % Common metrics for classification tasks, taken from \cite{sokolova2009systematic}.}
        \begin{tabularx}{\textwidth}{l c X}
            %\hline
            Measure & Formula & Interpretation \\
            \hline
            Accuracy & $\frac{tp+tn}{tp+fn+fp+tn}$ & Class agreement on both labels \\
            Balanced Accuracy & $\frac{1}{2}(\frac{tp}{tp+fn}+\frac{tn}{fp+tn})$ & Classifier’s ability to avoid false classification \\
            Precision & $\frac{tp}{tp+fp}$ & Class agreement with the positive labels given by the classifier \\
            Recall/Sensitivity & $\frac{tp}{tp+fn}$ & Effectiveness of a classifier to identify positive labels \\
            F$_{\beta}$-score & $\frac{(\beta^2+1)tp}{(\beta^2+1)tp+\beta^2fn+fp}$ & Harmonic mean between precision and recall with choosable $\beta$ \\
            Specificity & $\frac{tn}{fp+tn}$ & How effectively a classifier identifies negative labels \\
        \end{tabularx}
    \end{center}
    \label{tab:class_metrics}
\end{table}

A classifier can also be used to predict a class probability.
For the case of binary classification, this can be a number between 0 and 1,
with 0 being one class and 1 being the second.
The classifier threshold is then the value at which the separation into the two classes happens.
One can then calculate the Area Under the Curve (AUC) with regard to the Receiver Operating Characteristic (ROC) curve.
The ROC-curve is gained by plotting the true positive rate against the false positive rate while varying the classifier threshold.
The area under the (normalised) ROC-curve is thus a measure for whether the classifier will
rank a randomly chosen positive sample higher than a randomly chosen negative sample \cite{FAWCETT2006861}.

\subsubsection{Regression}
Regression is the task of predicting a continuous variable
from a set of input variables.
The simplest approach
to this problem is the ordinary linear least squares method.

Given an unrestricted linear model
\begin{align}
	y &= X\beta + e \\
	E(y) &= X\beta \\
	Cov(y) &= \sigma^2 I_n
\end{align}
with a measured vector $y$, the design matrix $X$,
an unknown parameter vector $\beta$, a random error $e$
and pairwise orthogonal features $y_i$,
the least-squares solution is given by the solution of
the minimizing problem in equation \ref{eq:min_least_squares}.

\begin{equation}
	\min_{\beta\in\mathbb{R}^k} \lVert y - X\beta \rVert
	\label{eq:min_least_squares}
\end{equation}

If $(X^TX)^{-1}$ exists, the unique solution for the
least square estimation of $\beta$ becomes:
\begin{equation}
	\hat{\beta} = X^+ y,
\end{equation}

with the Moore-Penrose inverse $X^+ = (X^TX)^{-1}X^T$.
The estimation of $y$ is then:
\begin{equation}
  \hat{y} = X\hat{\beta}.
\end{equation}

The metric, that is minimized by the least-squares solution
is the Mean Squared Error (MSE).

Other metrics for regression tasks include the
Root-Mean-Squared-Error(RMSE),
Mean-Absolute-Error(MAE)
or the Coefficient of Determination ($R^2$), all of which are listed in table
\ref{tab:regr_metrics}.

\begin{table}
  \caption{Popular metrics for a regression problem with $n$ samples.}
  \begin{center}
    \begin{tabularx}{\textwidth}{l c X}
      Measure & Formula & Interpretation \\
      \hline
      Mean squared error & $\frac{1}{n}\sum_i^n |y_i-\hat{y_i}|$ & Prediction error disregarding the direction of over- and underprediction \\
      Mean absolute error & $\frac{1}{n}\sum_i^n (y_i-\hat{y_i})^2$ & For an unbiased predictor: Variance of the regressor. Heavily weights outliers. \\
      Coefficent of Determination & $1 - \frac{\sum_i^n (\bar{y_i}-\bar{y})^2}{\sum_i^n (y_i-\bar{y})^2}$ & Share of observed variance that is explained by the model.\\
    \end{tabularx}
  \end{center}
  \label{tab:regr_metrics}
\end{table}

Many metrics closely connected to these metrics exist, such as the root mean squared error (RMSE)
or the mean absolute percentage error (MAPE).

\subsubsection{Decision Trees and Random Forests}
A simple (binary) decision tree
for an example dataset with 50 proton and 50 gamma
events is shown in figure \ref{fig:03_tree}.

\begin{figure}
  \centering
  \captionsetup{width=0.9\linewidth}
  \includegraphics[width=0.9\textwidth]{Plots/decision_tree.pdf}
  \caption{A decision tree for a small sample dataset consisting of 50 proton and
  50 gamma events. The tree is capped at a depth of 4 and splits only on the 
  features width, length and intensity. On the training data, this tree reaches an
  accuracy of \num{0.81}.}
  \label{fig:03_tree}
\end{figure}

Starting from the root node, a binary split is performed to
split up the data. For each resulting node, additional splits are performed
until a stopping criterion is reached.
Choosing the optimal split is defined as minimizing a
pre-defined measure.

For classification tasks this means reducing the class impurity in the node.
Often used measures
to quantify the impurity are gini coefficient or the
cross-entropy \cite{hastie2017springer}.
Both are defined in equation \ref{eq:gini_ce}

\begin{align}
	\text{Gini impurity: } &= \sum_{k=1}^K \hat{p}_{mk}(1-\hat{p}_{mk}) \\
	\text{Cross-entropy: } &= -\sum_{k=1}^K \hat{p}_{mk}\log{\hat{p}_{mk}},
  \label{eq:gini_ce}
\end{align}

with $p_{mk}$ denoting the proportion of class $k$ in node $m$.

A stopping criterion can be defined as the measure reaching a
a defined threshold or not improving anymore.
Alternatively the tree can stop at a predefined depth to
avoid overly complex models.

For regression tasks scikit-learn uses the mean squared error
or mean absolute error and the same principles apply.

The implementation in sklearn is based on the one by
Leo Breiman et al \cite{breimanclassification}.
A single tree performs binary splits $\Theta = (j, t_m)$
at each node $m$ in order to split
the data at this node $Q$ into two subsets
$Q_\text{left}$
and
$Q_\text{right}$.
The split consists of a feature $j$ and a threshold $t_m$ and is
chosen in a way to minimize the given measure.
Features, that are more important for the task, will
thus appear at the top nodes of the tree.


While decision trees have the benefit of providing
easily interpretable, low bias models there are some drawbacks to this
approach, namely \cite{hastie2017springer}:
\begin{itemize}
  \item{Instability, high variance}
  \item{Lack of Smoothness}
  \item{Difficulty in Capturing Additive Structure}.
\end{itemize}

One approach to reduce these problems is the construction of
random forests \cite{Breiman2001}.

The main idea behind random forests is to use multiple
decision trees to suppress the problems single trees have, while
keeping their advantages.
For this to work, the individual trees should  be correlated.
Consequently the trees cannot all be constructed the same way.
To make sure the individual trees
- and their predictions -
are somewhat independent from each other,
randomness has to be introduced to the construction of the tree.
In random forests this is on the one hand achieved by giving each tree a
random subsample from the training data, obtained with bootstrapping \cite{efron1992bootstrap}.
Another source of randomness is to perform splits on a node
based on only a random subsample of the available features.

The prediction of the random forest in scikit learn is then the average of
the single trees predictions.
In the case of a classification task, the probabilistic predictions for each class
get averaged.

% \chapter{analysis}\label{analysis}

ToDo:
\begin{enumerate}[nosep]
    \item Datenformat Input, verwendete Daten, Worte zu Simulation, trainingsdaten
    \item Preprocessing - Schritte erklären
    \item Datenformat Output Preprocessing
    \item ...
    \item Anwendung Random Forests mono, disp erklären anhand hillas parameter
    \item Anwendung Stereo (Random forest, stabile Mittelwertverfahren -alle erklären)
    \item Signifikanzkurve
\end{enumerate}

\section{Data Levels, preprocessing}

\subsection{DL0, simtel}
At the lowest level our observed data consists of uncalibrated waveforms 
in the camera pixels.

\subsection{From Dl0 to DL1}

- calibration
- cleaning
- hillas parameters 
- telescope level



\begin{figure}
    \begin{subfigure}{0.3\textwidth}
        \includegraphics[width=0.9\linewidth]{../Plots/hillas_raw.pdf} 
        %\caption{Caption1}
        \label{fig:shower_image_raw}
    \end{subfigure}
    \begin{subfigure}{0.3\textwidth}
        \includegraphics[width=0.9\linewidth]{../Plots/hillas_cleaned.pdf}
        %\caption{Caption 2}
        \label{fig:shower_image_cleaned}
    \end{subfigure}
    \begin{subfigure}{0.3\textwidth}
        \includegraphics[width=0.9\linewidth]{../Plots/hillas_cleaned_params.pdf} 
        %\caption{Caption1}
        \label{fig:hillas_parameters_only}
    \end{subfigure}
    \caption{Illustration of a simulated gamma shower captured with the LST-telescope.
        The left figure shows the image after applied waveform-extraction but before
        the cleaning step. The right figure shows the image after the cleaning has been applied
        and non-selected pixels have been discarded.}
    \label{fig:shower_image}
\end{figure}

\subsection{From DL1 to DL2}  % we dont seperate between 0 and 1 right?
erklären, dass hillas reco reinkommt -> stereo features

\subsection{Machine Learning, dl3?}
High level analysis of the preprocessed data is based on the use of
the aict-tools \cite{aict-tools} package which is based on
sklearn \cite{sklearn_api} for the machine learning algorithms.
The algorithm of choice is the Random Forest algorithm
as it is well suited for the use with tabluar data and tends to not overfit
(citation needed).
Seperate models are trained for the tasks of signal/background
separation, signal energy estimation and signal source position
reconstruction.
The aict-tools have originally been developed for the FACT-experiment
(citation needed) which is a single IACT. For this reason
different ways of combining the results from single telescopes events
to stereo events will be presented.


\section{g/h sep}
For the task of gamma/hadron separation a random forest can be trained
using either only monoscopic information or also using array-level
information from earlier reconstruction steps.
This generally improves the accuracy by a few percent.
The single telescope predictions can be combined by
simple functions such as the mean or median of the
single predictions to provide a prediction for the complete
array-event.
Alternatively the single telecope predictions can be used to
feed a second random forest to provide array-level predictions.
Such a second model can easily be trained on array-level
features, e.g. mean and std of the single predictions, number of triggered
telescopes, total luminosity or features calculated from the
Hillas-Reconstructor if not already used for the single telescope
prediction. in this case the model would effectively deviate
from the mean of the prediction according to information that the
single predictions did not include.


STEREO IN REKONSTRUKTION EINBAUEN??


\section{energy estimation}
Energy estimation can be performed in the same way as the gamma/hadron
separation. For this task there has been earlier work indicating
the usefulness of a second machine learning model trained
on the predictions of the first telesope-level model
\cite{ba-lars}.

I am thus going to present results based on either calculating the mean
of the telescope level predictions and using a second random forest
to improve the array level prediction.

\section{source position}
\label{sec:source_position}
Given the source position in the camera frame the source position
on the sky can be calculated with coordinate transformations if
the position, pointing and optical properties of the
telescope are known.
In general the true source position in the camera frame is assumed to be
different from the center of gravity of the shower ellipse
but located somewhere along the main shower axis.
The position on the shower axis can be estimated based on 
the hillas parameters and other image features.
This method is known as the DISP-method in the
literature (citation needed). The general idea 
can be seen in figure \ref{fig:disp}.

\begin{figure}
    \includegraphics[width=0.9\linewidth]{../Plots/hillas_complete.pdf}
    \caption{Illustration of monoscopic source position reconstruction making use of 
        the Hillas-Parameters and the DISP-method as explained in section \ref{sec:source_position}.
        The left figure has the hillas ellipse and parameters drawn onto our previously cleaned sample shower.
        The right figure estimates a source Position in the Camera frame.}
    \label{fig:disp}
\end{figure}

With the DISP-method the estimated distance between the source
position and the center of gravity of the hillas ellipse gets calculated
based on the form of the ellipse, timing information and potentially
more features.
This can be done analitically, via lookup-tables or with machine learning.
At this point the reconstructed source position
is fixed at two points at the main shower axis, see figure \ref{fig:disp_amb}


\begin{figure}
    \includegraphics[width=0.9\linewidth]{../Plots/hillas_complete.pdf}
    \caption{Wrong pic!}
    \label{fig:disp_amb}
\end{figure}


The remaining task then consists of finding the correct one of these
two points. If there is no stereoscopic information available,
a choice can be made based on the image features again.
In FACT-analyses a second random forest is trained for this
specific purpose, usually yielding an accuracy around 70-80\% (citation).
This is called SIGN-prediction, interpreting the two possible sides
as +-1.
In the case of the MAGIC-telescopes the ambiguity does not
get resolved until the individual results get combined
to the stereo level. The choice of the correct
pair out of the four reconstructed positions can be done either
by calculating the crossing point of both main shower axises
or by calculating the pairwise distances between the positions (citation).


These methods are illustrated in figure \ref{fig:disp_magic}


\begin{figure}
    \begin{subfigure}{0.3\textwidth}
        \includegraphics[width=0.9\linewidth]{../Plots/hillas_raw.pdf} 
        %\caption{Caption1}
        \label{fig:3}
    \end{subfigure}
    \begin{subfigure}{0.3\textwidth}
        \includegraphics[width=0.9\linewidth]{../Plots/hillas_cleaned.pdf}
        %\caption{Caption 2}
        \label{fig:2}
    \end{subfigure}
    \begin{subfigure}{0.3\textwidth}
        \includegraphics[width=0.9\linewidth]{../Plots/hillas_cleaned_params.pdf} 
        %\caption{Caption1}
        \label{fig:1}
    \end{subfigure}
    \caption{wrong pics}
    \label{fig:disp_magic}
\end{figure}



The analysis software ctapipe (citation) that gets developed for
CTA, currently implements a slightly different, geometric approach.
The so called HillasReconstructor constructs hillas planes
from the pointing of the telescopes and the reconstructed hillas parameters
and calculates a weighted average of the intersections.

A schematic illustration is given by figure \ref{fig:hillas_reco}

\begin{figure}
    \centering
    \includegraphics[width=0.8\textwidth]{./Plots/Histogramm.pdf}
    \caption{HillasReconstructor}
    \label{fig:hillas_reco}
\end{figure}

Obviously this requires at least two triggered telescopes.
On the other hand this approach scales
very well for high numbers of telescopes and
also provides an etimation of the interaction height.
Competitive performance is generally reached
at around 3 telescopes (citation).


A DISP-based method for CTA is required to perform a reconstruction on
events with an arbitrary number of triggered telescopes.
As this scaling can be somewhat nontrivial in some cases I am gonna
compare a few different approaches and compare the performances
with a single telescope DISP-prediction and the HillasReconstructor.
Calculation of the DISP and SIGN at
telescope-level is performed using the aict-tools(citation).
In the background random forest are constructed for both the
DISP-regression and the SIGN-classification.
For combining the individual predictions to an array level prediction
the following methods will be compared:

\textbf{mean}
The correct estimation is deciced at telescope-level usign the
SIGN-information. The mean of all positions gets calculated as
array prediction.
This implicitly assumes all SIGN's are correctly classified as
wrong classification leads to very far off predictions.
As this is not the case and accuracies >80\% are not realistic,
this naive approach is expected to perform not better than the individual
telescope predictions.
It is mainly included as a benchmark to emphasize the problems
that follow the reconstruction ambiguity and are not present for
energy regression or g/h separartion.


\textbf{outlier resistant average estimation}
%\textbf{median}
Assuming that most events get classified correctly, a calculation of
the average of all telescope-predictions might still be successful if
events with wrong reconstructed SIGN's are not taken into account.
The simplest way to perform an outlier resistant reconstruction
would be to take the median instead of the mean.
Another possible algorithmus iteratively calculates
the median and and standard deviation of the cluster and removes
outliers based on their distance to the median.
This algorithmus is implemented as "sigma_clipped_stats" in astropy (citation).

\textbf{Pairwise combination (MAGIC-based)}
Since algorithms for choosing the correct points exist for the
case of two telescopes, one could treat each pair of observations 
the same way the magic telescopes do to resolve the head-tail-ambiguity.
The results of all of these pairs could then be combined by taking the mean 
or median.
The following figure \ref{fig:disp_cta_magic} illustrates the general idea of iteratively 
choosing two of the triggered telescopes.

\begin{figure}
    \begin{subfigure}{0.3\textwidth}
        \includegraphics[width=0.9\linewidth]{../Plots/stereo_magic_1.pdf} 
    \end{subfigure}
    \begin{subfigure}{0.3\textwidth}
        \includegraphics[width=0.9\linewidth]{../Plots/stereo_magic_2.pdf}
    \end{subfigure}
    \begin{subfigure}{0.3\textwidth}
        \includegraphics[width=0.9\linewidth]{../Plots/stereo_magic_3.pdf} 
    \end{subfigure}
    \begin{subfigure}{0.3\textwidth}
        \includegraphics[width=0.9\linewidth]{../Plots/stereo_magic_4.pdf} 
    \end{subfigure}
    \begin{subfigure}{0.3\textwidth}
        \includegraphics[width=0.9\linewidth]{../Plots/stereo_magic_5.pdf}
    \end{subfigure}
    \begin{subfigure}{0.3\textwidth}
        \includegraphics[width=0.9\linewidth]{../Plots/stereo_magic_6.pdf} 
    \end{subfigure}
    \caption{Its a king of magic}
    \label{fig:disp_cta_magic}
\end{figure}

- idea
- hope


\textbf{HESS-method}
Das ist mega nice, aber braucht error estimates.

\textbf{Veritas-method}
Since the predictiosn with the correct sign should cluster around the true source
position and the misclassified predictions should be less densely clustered, we 
can formulate our problem as: Find the biggest cluster of points and average these.
This sounds like a textbook application of clustering algorithms.
The algorithm of choice for this work has been the DBSCAN algorithm (citation)
as implemented in sklearn (citation). The algorithm requires a minimal distance to 
be defined for points to count as neighbours. It then collects clusters and 
treats non associated points as background. We then take the cluster with the most 
points in it and average over these points.

The choice of a minimal distance directly influences the performance 
of the algorithm as too small of a distance leads to points left out and potentially 
wrong identification of the main cluster as clusters get small while 
choosing the mininmal distance too big includes misclassified predictions 
making the averaging harder.

\begin{figure}
    \includegraphics[width=0.9\linewidth]{../Plots/hillas_complete.pdf}
    \caption{Wrong pic!}
    \label{fig:disp_amb}
\end{figure}


\begin{figure}
    \includegraphics[width=0.9\linewidth]{../Plots/hillas_complete.pdf}
    \caption{Wrong pic!}
    \label{fig:disp_amb}
\end{figure}


\section{Analysed Data}
\subsection{Corsika Simulation}
\subsection{Training, Testing, mismatches und so, weniger telescope daten bei weniger teleskopen! duh... aber wichtig für statistik.}

% \chapter{results}\label{results}

\section{ghsep}\label{ghsep}
The trained random forests reaches an area under the ROC-curve (???) 
of XXXX, as can be seen in figure \ref{fig:gh_roc}.

\begin{figure}
    \includegraphics[width=.8\textwidth]{Plots/decision_tree.pdf}
    \caption{ROC for the gamma/hardon seperation on the cross validated training set 
    consisting of XXX proton events and YYY diffuse gamma events.}
    \label{fig:gh_roc}
\end{figure}

The feature importance, as defined in chapter XXX, is shown in \ref{fig:gh_features}.
\begin{figure}
    \includegraphics[width=.8\textwidth]{Plots/decision_tree.pdf}
    \caption{Feature importance for the gamma/hardon separation.}
    \label{fig:gh_features}
\end{figure}


Applied to our clean testset of pointlike gammas, we get:
.....
sowas wie der zweite plot da machen.


% \section{position}\label{position}

FEATURE IMPORTANCE FOR THE SIGN MODEL

\subsection{Analysis at telescope level}

The random forests for the DISP- and SIGN-prediction get trained on
XXX diffuse gamma events with a 3-fold cross-validation.
The SIGN-model is not needed for this part, but can be useful as 
a cross-check for the LST-mono analysis in section \ref{lst mono part}.

The performance of the models can be gauged by looking at the 
figures \ref{fig:disp_test_perf} for the cross-validated set and 
\ref{fig:disp_test_perf} for the performance our 
pointlike data. Be aware that due to the chosen binning (constant bin width),
the bins do not necessarily contain a comparable amount of events.
This is especially relevant for the highest energy bins.

In both cases we can see that the DISP-model gets rapidly more
powerful with higher energies up until ~\SI{1}{\tera\electronvolt} from 
where on the performance does not improve anymore and in fact seems to decline
again. 

For the diffuse test set, a dip around \SI{3}{\tera\electronvolt}
up to \SI{70}{\tera\electronvolt} can be made out.
This seems to be independent of the binning, as equally filled bins 
do not change this behaviour, see figure \ref{fig:disp_test_perf_2}.

For pointlike gamma events the performance seems to decrease as well, but does not 
recover at the highest energies.
This effect is less prominent with equally filled bins for the pointlike
gammas, see figure \ref{fig:disp_gamma_perf_2}.

The SIGN-model does not saturate like this and improves pretty linear up until
the highest energies. The optimal performance on diffuse events seems 
to be capped earlier than on pointlike events.

\begin{figure}
    \begin{subfigure}{0.45\textwidth}
        \includegraphics[width=0.9\linewidth]{../analysis/plots/disp_test_r2_equal_sized.pdf} 
        \caption{R2-Score for the DISP-estimation}
    \end{subfigure}
    \begin{subfigure}{0.45\textwidth}
        \includegraphics[width=0.9\linewidth]{../analysis/plots/disp_test_acc_equal_sized.pdf}
        \caption{SIGN-accuracy}
    \end{subfigure}
    \caption{Performance of the DISP- and SIGN-estimation algorithm on the test-dataset.}
    \label{fig:disp_test_perf}
\end{figure}

\begin{figure}
    \begin{subfigure}{0.45\textwidth}
        \includegraphics[width=0.9\linewidth]{../analysis/plots/disp_gamma_r2_equal_sized.pdf} 
        \caption{R2-Score for the DISP-estimation}
    \end{subfigure}
    \begin{subfigure}{0.45\textwidth}
        \includegraphics[width=0.9\linewidth]{../analysis/plots/disp_gamma_acc_equal_sized.pdf}
        \caption{SIGN-accuracy}
    \end{subfigure}
    \caption{Performance of the DISP- and SIGN-estimation algorithm on the pointlike dataset..}
    \label{fig:disp_gamma_perf}
\end{figure}

\begin{figure}
    \begin{subfigure}{0.45\textwidth}
        \includegraphics[width=0.9\linewidth]{../analysis/plots/disp_test_r2_equal_filled.pdf} 
        \caption{R2-Score for the DISP-estimation}
    \end{subfigure}
    \begin{subfigure}{0.45\textwidth}
        \includegraphics[width=0.9\linewidth]{../analysis/plots/disp_test_acc_equal_filled.pdf}
        \caption{SIGN-accuracy}
    \end{subfigure}
    \caption{Performance of the DISP- and SIGN-estimation algorithm on the test-dataset.}
    \label{fig:disp_test_perf_2}
\end{figure}

\begin{figure}
    \begin{subfigure}{0.45\textwidth}
        \includegraphics[width=0.9\linewidth]{../analysis/plots/disp_gamma_r2_equal_filled.pdf} 
        \caption{R2-Score for the DISP-estimation}
    \end{subfigure}
    \begin{subfigure}{0.45\textwidth}
        \includegraphics[width=0.9\linewidth]{../analysis/plots/disp_gamma_acc_equal_filled.pdf}
        \caption{SIGN-accuracy}
    \end{subfigure}
    \caption{Performance of the DISP- and SIGN-estimation algorithm on the pointlike dataset..}
    \label{fig:disp_gamma_perf_2}
\end{figure}


Having a look at figure \ref{fig:disp_features} we can learn that 
the stereoscopic features "$distance_to_reconstructed_core_position$"
and $h_max$ have a high influence on the DISP-prediction.
Apart from that the most influence seems to come from the constructed 
feature log\_size\_area, which roughly describes how much light per 
area hit the camera. The timing parameters seem to provide information as 
well. The model also learns on the features focal\_length,  
camera\_type\_id and telescope\_type\_id. This is expected 
behaviour as we have different types of telescopes in the datasets.

\begin{figure}
	\centering
	\includegraphics[width=0.8\textwidth]{../analysis/plots/disp_features.pdf}
	\caption{disp features}
	\label{fig:disp_features}
\end{figure}

The final results can be seen in figure \ref{fig:sens_telescope}.
One can derive - in accordance to the previously discussed metrics - 
that the DISP-prediction improves with increasing energies and there are less
missclassified SiGNs at higher energies.
We can also see the saturated DISP-performance at \SI{1}{\tera\electronvolt}.

\begin{figure}
    \centering
    \includegraphics[width=.8\textwidth]{../analysis/plots/gamma/tel_vs_energy.pdf}
    \caption{Per telescope predictions for the source position. The blue-dotted line 
    shows the 68\% containment. A lot of misclassified events at the lower energies
    lead to a second population above the main one below \SI{1}{\degree}.
    The misclassification rate seems to decrease with higher energies 
    in accordance to figures \ref{fig:disp_gamma_perf} and \ref{fig:disp_gamma_perf_2}.
    For very high energies the predictions do not improve despite the lower 
    misclassification rate. This also fits our earlier conclusions.}
    \label{fig:sens_telescope}
\end{figure}


\subsection{Analysis for the stereoscopic array}
MEDIAN NUTZT SIGN -> VORHER KORRIGIEREN UND HIER ERWÄHNEN

As a baseline comparison we compare the median DISP-prediction 
with the results obtained with the Hillas-reconstructor.
Choosing the median should get rid of some misclassification 
problems as events with more correctly than incorrectly 
predicted SIGNs will lead to decent predictions.
These results can be seen in figure \ref{fig:stereo_double_median}
with the 68\% containment of the HillasReconstructor in green 
and the median predictions in blue.
The 2d-histogram in the back refers to the median DISP-predictions,
the HillasReconstructor predictoins are not shown (other than the blue line).

At each energy the Hillas-reconstructor performs considerably better.
One can see that the median predictions do not improve considerably after
\SI{3}{\tera\electronvolt} and gets worse after \SI{20}{\tera\electronvolt}.
The Hillas-reconstructor shows a similar bump at the highest energies.
Due to the low statistics in these regions we can not really tell 
if this shows the limitations of the telescopes sensitivity range 
or if there are just some hard-to-reconstruct events in there. 

The HillasReconstructor also shows a short bump in the range of \SI{200}{\giga\electronvolt}
to \SI{1}{\tera\electronvolt}. 
This has been observed earlier and seems to correlate with 
a lot of low multiplicity events in the crossover region from LSTs to MSTs \cite{????}.

Looking at the multiplicity we can conclude that the median predictions
do not scale as well with higher multiplicities.

\begin{figure}
    \centering
    \begin{subfigure}{0.5\textwidth}
        \includegraphics[width=\linewidth]{../analysis/plots/gamma/median_vs_energy.pdf} 
        \caption{Distance to true position against energy}
    \end{subfigure}
    \begin{subfigure}{0.5\textwidth}
        \includegraphics[width=\linewidth]{../analysis/plots/gamma/median_vs_multi_comp.pdf}
        \caption{Distance to true position against event multiplicity}
    \end{subfigure}
    \caption{Performance of the median telescope prediction compared 
    against the results of the HillasReconstructor. 
    Left: Binned results from the median DISP-predictions. The blue line shows the 
    68\% containment of these results. The green line refers to the 
    results of the HillasReconstructor which are not shown otherwise.
    Right: Performance against event multiplicity. Both algorithms requires 
    multiplicities $\geq 2$. Higher multiplicity events are cut off, because 
    statistic is low and the emphasis lies on the lower multiplicity events.
    Blue and green refer to the median predictions and the HillasReconstructor predictions.
    The different lines refer to the 25,50,68 and 90\% percentiles with 
    lowering alpha.
    The median predictions are worse at every multiplicity.}
    \label{fig:stereo_double_median}
\end{figure}

The results of the more sophisticated DISP-approach can be seen in figure \ref{fig:stereo_double_magic}.
Compared to figure \ref{fig:stereo_double_median} the completely misreconstructed events are almost
gone and the 68\% percentile is much improved throughout the complete energy range besides 
the very highest energies. At this point the DISP-error is probably limiting and the 
Hillas-reconstructor leads to much better results.
At the lower energy range our approach seems to be working pretty well, 
slightly outperforming the Hillas-reconstructor, especially 
where the HillasReconstructor shows bumps.

When looking at the multiplicity-plot, we see a similar picture as before but with 
better results. At high event multiplicities $\ge 36$ the Hillas-reconstructor 
leads to better results. Improvements can be seen especially at 2- and 3-multiplicity 
events. In the case of 2 telescopes the iterative approach is identical to the 
Magic-method we based the implementation on.
The Hillas-reconstructor on the other hand does not work too well with 2 
triggered telescopes, with 3 triggered telescopes leading to much better results 
already especially at the 90\% percentile.


\begin{figure}
    \centering
    \begin{subfigure}{0.5\textwidth}
        \includegraphics[width=\linewidth]{../analysis/plots/gamma/pairwise_median_100_vs_energy.pdf} 
        \caption{Distance in degree against energy}
    \end{subfigure}
    \begin{subfigure}{0.5\textwidth}
        \includegraphics[width=\linewidth]{../analysis/plots/gamma/pairwise_median_100_vs_multi_comp.pdf}
        \caption{Distance in degree against multiplicity}
    \end{subfigure}
    \caption{Performance of the combined DISP-predictions as per our iterative approach
    compared against the results of the HillasReconstructor.
    Left: Binned results from the iteratively combined DISP-predictions.
    The blue line shows the 
    68\% containment of these results. The green line refers to the 
    same results of the HillasReconstructor as in figure \ref{fig:stereo_double_median} 
    which are not included in the binned results.
    Right: Performance against event multiplicity. Both algorithms requires 
    multiplicities $\geq 2$. Higher multiplicity events are cut off, because 
    statistic is low and the emphasis lies on the lower multiplicity events.
    Blue and green refer to the median predictions and the HillasReconstructor predictions.
    The different lines refer to the 25,50,68 and 90\% percentiles with 
    lowering alpha.
    For low multiplicities the combined DISP-predictions are superiour, 
    for high multiplicities the HillasReconstructor results win out.
    The breakeven point seems to be at 4-6 telescopes, depending on 
    which percentiles have more weight to the analysis.}
    \label{fig:stereo_double_magic}
\end{figure}


\section{Hadroness Cut}

In a analysis of real data, some legitimate events usually get discarded
in the gamma-/hadron-separation step.
Assuming that the misclassified events are in some way non regular, it might be 
justified to assume that these events also behave abnormal during the reconstruction of 
the source position. Discarding these events could thus improve the overall 
predictions.

Based on the performance of the gamma-/hadron-separation model (see figure \ref{add that earlier})
we set the gammaness cut at XXX.
This is done purely based on heuristics and does not involve any detailed analysis.

With this cut XXX events or XXX\% of the events remain.

We apply the same model as earlier and compare the results to the HillasReconstructor
again. IRGENDWIE QUANTIFIZIEREN -> THETA**2 VERTEILUNGEN FÜR ALLE ENERGIEN PLOTTEN UND 68\% ANSCHAUEN?


\begin{figure}
    \centering
    \begin{subfigure}{0.9\textwidth}
        \includegraphics[width=\linewidth]{../analysis/plots/gamma_cut/pairwise_median_100_vs_energy.pdf} 
        \caption{Distance in degree against energy}
    \end{subfigure}
    \begin{subfigure}{0.9\textwidth}
        \includegraphics[width=\linewidth]{../analysis/plots/gamma_cut/pairwise_median_100_vs_multi_comp.pdf}
        \caption{Distance in degree against multiplicity}
    \end{subfigure}
    \caption{Performance of the superduper compared 
    against the Hillas-Reconstructor. Median in Blue, Hillas in green
    Actually better, but in the extreme regions where statistics is low as well.}
    \label{fig:stereo_double}
\end{figure}


-> bringt nichts?


\section{LST only}
To simulate early operation stages where the LST1 is the only fully functional 
telescope, the analysis is limited to only the single LST with 
the telescope\_id 4.

This means that we do not have a baseline Hillas Reconstructor as we can not use
stereoscopy and each array event holds exactly one telescope event.
We expect to
get a similar sensitivity as in the earlier figure \ref{fig:sens_telescope}.

We also try to limit the training data on this particular telescope.
This will leave us with far less training samples, but hopefully
a better model?????

HOW DOES THE PERFORMANCE CHANGE?
THIS REQUIRES MORE EVENTS AND MAYBE MORE LSTS FOR TRAINING?
COMPARE TO APPLYING THE NORMAL MODEL

\begin{figure}
    \begin{subfigure}{0.45\textwidth}
        \includegraphics[width=0.9\linewidth]{../analysis/plots/disp_test_mono_lst_r2_equal_filled.pdf} 
        \caption{R2-Score for the DISP-estimation}
    \end{subfigure}
    \begin{subfigure}{0.45\textwidth}
        \includegraphics[width=0.9\linewidth]{../analysis/plots/disp_test_mono_lst_acc_equal_filled.pdf}
        \caption{SIGN-accuracy}
    \end{subfigure}
    \caption{Performance of the DISP- and SIGN-estimation algorithm on the test-dataset.}
    \label{fig:disp_test_perf}
\end{figure}

\begin{figure}
    \begin{subfigure}{0.45\textwidth}
        \includegraphics[width=0.9\linewidth]{../analysis/plots/disp_gamma_mono_lst_r2_equal_filled.pdf} 
        \caption{R2-Score for the DISP-estimation}
    \end{subfigure}
    \begin{subfigure}{0.45\textwidth}
        \includegraphics[width=0.9\linewidth]{../analysis/plots/disp_gamma_mono_lst_acc_equal_filled.pdf}
        \caption{SIGN-accuracy}
    \end{subfigure}
    \caption{Performance of the DISP- and SIGN-estimation algorithm on the pointlike dataset..}
    \label{fig:disp_gamma_perf}
\end{figure}

Looks decent, energy obviously limited bc LST.

% 
\chapter{conclusion}\label{conclusion}

ToDo:
\begin{enumerate}[nosep]
    \item ???
    \item ausbblick: andere lerner? adaboost?
\end{enumerate}


\chapter{conclusion2}


% \newpage
% \appendix
% % Hier beginnt der Anhang, nummeriert in lateinischen Buchstaben
% \chapter{Parameters}
\label{sec:app_params}

config kram hier hin


\backmatter
\printbibliography



% \cleardoublepage
% \input{content/eid_versicherung.tex}
\end{document}
